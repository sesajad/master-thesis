 
\chapter{Discussion}\label{chap:discussion}

While remote gates and ...

It is a well-known fact that a SWAP gate could not be decomposed to less than three CNOTs. For a more complex circuit involving a SWAP or more, this does not necessarily hold in general, especially in presence of connectivity contraints. Here we provide two examples (\ref{ex:swap-swap-decomposition} and \ref{ex:cnot-bridge-decomposition}) with linear connectivity constraints of SWAPs that are used to apply a two-qubit gate on non-adjacent qubits.
They show that decomposing SWAPs to three CNOTs will not necessarily lead to an optimal circuit, even in simple cases.

\begin{figure}[ht]
  \centering
  \begin{tikzpicture}
  \node[scale=0.7] {
  \begin{quantikz}
    \qw & \swap{2} & \qw \midstick[3,brackets=none]{=} & \swap{1} & \qw & \swap{1} & \qw \midstick[3,brackets=none]{=} & \ctrl{1}&\targ{}&\ctrl{1} & \qw&\qw&\qw & \ctrl{1}&\targ{}&\ctrl{1} & \qw  \\
    \qw & \qw & \qw & \swap{} & \swap{1} & \swap{} & \qw & \targ{}&\ctrl{-1}&\targ{} & \ctrl{1}&\targ{}&\ctrl{1} & \targ{}&\ctrl{-1}&\targ{} & \qw \\
    \qw & \swap{} & \qw & \qw & \swap{} & \qw & \qw & \qw&\qw&\qw & \targ{}&\ctrl{-1}&\targ{} & \qw&\qw&\qw & \qw \\
  \end{quantikz} };
  \end{tikzpicture} \\

  \begin{tikzpicture}
    \node[scale=0.7] {
    \begin{quantikz}
      \qw & \swap{2} & \qw \midstick[3,brackets=none]{=} & \ctrl{1} & \qw &\targ{}& \qw & \ctrl{1}& \qw &\targ & \qw & \qw & \qw \\
      \qw & \qw & \qw & \targ{} & \targ{} & \ctrl{-1} & \ctrl{1} & \targ{} & \targ{} & \ctrl{-1} & \ctrl{1} & \qw \\
      \qw & \swap{} & \qw & \qw & \ctrl{-1} & \qw &\targ{}& \qw & \ctrl{-1}& \qw &\targ & \qw & \qw \\
    \end{quantikz} };
  \end{tikzpicture}
  \caption{Simplifying three SWAP gates}
  \label{ex:swap-swap-decomposition}
\end{figure}

\begin{figure}[ht]
  \centering
  \begin{tikzpicture}
  \node[scale=0.7] {
  \begin{quantikz}
    \qw & \ctrl{2} & \qw \midstick[3,brackets=none]{=} & \swap{1} & \qw & \swap{1} & \qw \midstick[3,brackets=none]{=} & \ctrl{1} & \targ{} & \ctrl{1} & \qw &\ctrl{1} & \targ{} & \ctrl{1} & \qw \\
    \qw & \qw & \qw & \swap{} & \ctrl{1} & \swap{} & \qw & \targ{} & \ctrl{-1}& \targ{} & \ctrl{1} & \targ{} & \ctrl{-1}& \targ{} & \qw \\
    \qw & \targ{} & \qw  & \qw & \targ{} & \qw & \qw & \qw & \qw & \qw & \targ & \qw & \qw & \qw & \qw  & \qw \\
    \end{quantikz}
  };
  \end{tikzpicture} \\
  \begin{tikzpicture}
    \node[scale=0.7] {
    \begin{quantikz}
    \qw & \ctrl{2} & \qw \midstick[3,brackets=none]{=} & \qw & \ctrl{1} & \qw & \ctrl{1} & \qw \\
    \qw & \qw & \qw & \ctrl{1} & \targ{} & \ctrl{1}  & \targ{} & \qw \\
    \qw & \targ{} & \qw & \targ{} & \qw  & \targ & \qw  & \qw &  \qw \\
    \end{quantikz}
    };
  \end{tikzpicture}

  \caption{Simplifying a CNOT bridge}
  \label{ex:cnot-bridge-decomposition}
\end{figure}

In order to generalize this idea, we define the following problem

\begin{problem}[Bridging two-qubit gates]
  The solution to the problem of Bridging two-qubit gate $U$ over $n$ qubits is a circuit that can be applied on $n + 2$ qubits such that the $n$ qubits in the middle are not affected and the two qubits on the ends are mapped with respect to $U$. The circuit should normally obey a connectivity and a gate set constraints.
\end{problem}

Note that this definition is valid for classical case as well. In this case, the gate $U$ is a classical reversible gate applying on bits instead of qubits.

% Noting that remote CNOTs and similar things have been already discovered. We will show that the problem of bridging two-qubit gates is a generalization of remote CNOTs and similar things.

\section{Classical Bridging}\label{sec:classical_bridging}

Quantum logical circuits or classical reversible circuits are a class of circuits that are in common between classical and quantum computation. This class has been defined carefuly \cite{shende2003} and it has been proven that any classical reversible circuit can be synthesized using $X$ (classically known as NOT), CNOT and TOFFOLI gates.

Two-bit classical gates are limited to identity, CNOT (in both directions), CNOT times CNOT in reverse direction, and finally SWAP; all upto local isomorphism (NOT gates).

We will show that CNOT could be bridged over $n$ qubits using $4n + O(1)$ CNOTs within the depth of $n + O(1)$. Then, we can prove that not only this is result is optimal, but also SWAP has an optimal bound $6n + O(1)$ CNOTs and $3n + O(1)$ depth that is already fulfilled by a simple decomposition.

In order to analyse the problem, we need to define a few concepts. We know that a classical gate is a $\{0, 1\}^n \to \{0, 1\}^n$ function. Then, we can define a notion of dependency like below

\begin{definition}[Classical dependency]
  Assume a binary function $f: \{0, 1\}^n \to \{0, 1\}^n$ say that output bit $j$ depends on $i$ if exist $x$ and $y$ where $x_i \neq y_i$ and $x_k = y_k$ for all $k \ne i$ that satisfies $f(x)_j \neq f(y)_j$.
\end{definition}



\begin{theorem}
  A bridging CNOT applying on qubits enumerated from $1 \dots n + 2$ ($a$, $b$) over $n$ bits needs at least $4n$ CNOTs, where $n + 1$ must be the minimum distance between the two bits ($a$, $b$).
  \label{thm:bridging-cnot}
\end{theorem}
\begin{proof}
  We can layer the graph of bits in a way that the first layer is $a$, the last layer is $b$ and there are $n$ layers that we name any bit in layer $i$ as $c_i$. We also define right as the direction towards $b$ and left as th opposite direction.

  By considering a $\mathbb{F}_2$ vector space with hamming inner product $\langle \cdot, \cdot \rangle$ over the bits,  while each bit at each time has a unique vector, we can see that each CNOT as a linear operation adding a vector to another.

  If we name the set of every vector that is associated with a CNOT from layer $i$ to layer $i+1$ with $F_{i\rightarrow i+1}$ and simliarly $F_{i\leftarrow i+1}$, from the fact that the vector $a$ is going to be added to $b$ at the end, we know that the vector $a$ could be extracted from $F_{i\rightarrow i+1}$, which means that $a \in \mathrm{span}(F_{i\rightarrow i+1})$. On the other hand, in order to layer $i$ be unchanged after the process $\sum_{F_{i\leftarrow i+1}} f + \sum_{F_{i-1 \rightarrow i}} f = 0$.
  Mathematically, we can conclude from these two fact that one of these two cases hold for each layer $i$:

  Case 1 ($P(i)$): $F_{i\rightarrow i+1} = D \cup D'$ where $D \cap D' = \emptyset$ and $a \in \mathrm{span}(D)$ and $a \in \mathrm{span}(D')$.

  Case 2 ($Q(i)$): exists $f$ that $\langle f, a\rangle = 1$ and $f \in F_{i+1 \leftarrow i+2}$. 

  We separately prove that for the first case 
  \begin{equation}
    \abs{F_{i \rightarrow i+1}} + \abs{F_{i-1 \leftarrow i}} \geq 4
  \end{equation}
  and for the second case
  \begin{equation}
    \abs{F_{i+1 \rightarrow i+2}} + \abs{F_{i+1 \leftarrow i+2}} \geq 4
  \end{equation}

  Moreover we prove that $P(k+1)$ and $Q(k)$ cannot happen at the same time. This means that if $Q(k)$ holds, $Q(j)$ holds for all $j \geq k$.

  Then assuming $Q(k)$ and $P(k-1)$, by summing up the bounds

  \begin{equation}
    \sum_{i=0}^{k-1} \abs{F_{i \rightarrow i+1}} + \abs{F_{i-1 \leftarrow i}} + \sum_{i=k}^{n-2} \abs{F_{i+1 \rightarrow i+2}} + \abs{F_{i+1 \leftarrow i+2}} \geq 4n - O(1)
  \end{equation}

  This is due to the fact that these two summations has $O(1)$ terms in common. This means that the total number of CNOTs is at least $4n + O(1)$.
\end{proof}

\begin{lemma}[Bound for $P(i)$]
  For the first case used in the proof of Theorem~\ref{thm:bridging-cnot}, we have 
  \begin{equation}
    \abs{F_{i \rightarrow i+1}} + \abs{F_{i-1 \leftarrow i}} \geq 4
  \end{equation}
\end{lemma}
\begin{proof}
  If we call any arbitrary bit in $i$th layer by $c_i$, we know if $a \in F_{i \rightarrow i + 1}$,then it means that there was CNOTs, involving in removing $c_i$ from its site and returning it back. By a simple argument of capacity (in bits) the data could only be stored in the left-hand-side of $i$ meaning that vectors with $c_i$ element appeared in each of $F_{i-1 \leftarrow i}$ $F_{i-1 \rightarrow i}$ twice.

  Now, we already assumed that $a \in \mathrm{span}(D)$ and $a \in \mathrm{span}(D')$. 

  Case 1a: If $a \in D$ or $a \in D'$, then $a \in F_{i \rightarrow {i+1}}$. From the argument above we know that the vectors containing $c_i$ appear twice in $F_{i-1 \leftarrow i}$. One could easily conclude that $\abs{F_{i \rightarrow i+1}} + \abs{F_{i-1 \leftarrow i}} \geq 4$. 

  Case 1b: If $a \notin D$ and $a \notin D'$, then both $D$ and $D'$ need to have two vectors to have $a$ in their span, meaning $\abs{F_{i \rightarrow i+1}} \geq 4$.
\end{proof}

\begin{lemma}[Bound for $Q(i)$]
  For the second case used in the proof of Theorem~\ref{thm:bridging-cnot}, we have 
  \begin{equation}
    \abs{F_{i+1 \rightarrow i+2}} + \abs{F_{i+1 \leftarrow i+2}} \geq 4
  \end{equation}
\end{lemma}
\begin{proof}
  We know that $f \in F_{i+1 \rightarrow i+2}$ that $\langle f, a \rangle = 1$, in addition to that, $
  a \in \mathrm{span}(F_{i \rightarrow i+1})$ and independently $a \in \mathrm{span}(F_{i+1 \leftarrow i+2})$.

  If $\langle f, c_{i+2} \rangle = 0$ it means that $c_{i+2}$ must appear twice in each $F_{i+1 \rightarrow i+2}$ and $F_{i+1 \leftarrow i+2}$, so $F_{i+1 \rightarrow i+2} + F_{i+1 \leftarrow i+2} \geq 4$.

  Otherwise if $\langle f, c_{i+2} \rangle = 1$, then $c_{i+2}$ appears once in $F_{i+1 \leftarrow i+2}$. Noting the order of CNOTs, it must be before CNOTs associated with $a$, therefore all of the terms in $F_{i+1 \rightarrow i+2}$ with $a$ must have $c_{i+2}$ component as well. This makes impossible to $a \in F_{i+1 \rightarrow i+2}$. So $a \in \mathrm{span}(F_{i+1 \rightarrow i+2})$ results in $a$ appears twice, so $\abs{F_{i+1 \rightarrow i+2}} \geq 2$. Putting all together, $\abs{F_{i+1 \rightarrow i+2}} + \abs{F_{i+1 \leftarrow i+2}} \geq 4$.
\end{proof}

It could be easily shown that the circuit that is visualized in Figure \ref{fig:optimal-bridging-cnot} is an optimal solution for bridging CNOT.

\begin{figure}[ht]
  \centering
  \begin{tikzpicture}
    \node[scale=0.7] {
      \begin{quantikz}
        \qw &\ctrl{5}&\qw \midstick[6,brackets=none]{=} &\targ{}  & \qw     &\ctrl{1}& \qw    & \qw    & \qw    &\ctrl{1}& \qw     &\targ{}&\qw\\
        \qw & \qw    & \qw    &\ctrl{-1}&\targ{}  & \targ{}&\ctrl{1}& \qw    &\ctrl{1}&\targ{} &\targ{}  &\ctrl{-1}&\qw\\
        \qw & \qw    & \qw    & \qw     &\ctrl{-1}& \qw    & \targ{}&\ctrl{1}&\targ{} & \qw    &\ctrl{-1}&\qw & \qw \\
        \qw & \qw    & \qw    & \qw     &\targ{}  & \qw    &\ctrl{1}& \targ{}&\ctrl{1}& \qw    &\targ{}  &\qw & \qw\\
        \qw & \qw    & \qw    &\targ{}  &\ctrl{-1}&\ctrl{1}& \targ{}& \qw    &\targ{} &\ctrl{1}&\ctrl{-1}&\targ{}&\qw \\
        \qw &\targ{} & \qw    &\ctrl{-1}& \qw     & \targ{}& \qw    & \qw    & \qw    &\targ{} & \qw     &\ctrl{-1}& \qw 
        \end{quantikz} };
  \end{tikzpicture}
  \caption{Optimal bridging CNOT}
  \label{fig:optimal-bridging-cnot}
\end{figure}

TODO: Show that two TOFFOLIs (with a NOT in between) are equal to a bridged CNOT over one bit.

TODO: Prove the optimality in presence of TOFFOLI in gate set.

\begin{theorem}
  Bridging SWAP($a$, $b$) over $n$ bits needs at least $6n$ CNOTs, where $n + 1$ must be the minimum distance between the two bits ($a$, $b$).
  \label{thm:bridging-swap}
\end{theorem}
\begin{proof}
  TODO
\end{proof}

\section{Quantum Bridging}

\begin{lemma}
  Any two-qubit gate with Schmidt number $1$ could be written as $L_1 \otimes L_2 CR_x(\theta) L_3 \otimes L_4$ where $L_i$ are local unitaries and $CR_x(\theta)$ is a controlled $R_x$ gate.
  \label{lem:decomposition-schmidt-2}
\end{lemma}
\begin{proof}
  Any two-qubit gate with Schmidt number $2$ could be written as $L_1 \otimes L_2 e^{i\alpha ZZ} L_3 \otimes L_4$. 
  Ignoring the local operations, by applying $R_z(-\theta)$ on the first and the second qubit, we will have 
  \begin{equation}
    R_z(-\theta) \otimes R_z(-\theta) e^{i\alpha ZZ} = CR_z(3\theta)
  \end{equation}
  
  that could be converted to $CR_x(\theta)$ by applying $H$ on the second qubit.
\end{proof}


\begin{theorem}
  Any two-qubit gate with Schmidt number $1$ can be bridged over $n$ qubits using $4n + O(1)$ CNOTs
\end{theorem}
  
\begin{proof}
  We know that this could be easily done by swapping the first qubit one-by-one up to the middle
\end{proof}

\subsection{Optimality}

Definition:

$U$ acts trivial on $i$ if we can say $U = I_i \otimes U'$ for some $U'$, otherwise $U$ acts non-trivial on $i$.

Proof snippet

- $X_i L = L U_i$

- $X_i CNOT_{i,j} = CNOT_{i,j} X_i X_j$

- $Z_j CNOT_{i,j} = CNOT_{i,j} Z_i Z_j$

- exists $U$ and $U'$ that $U_i CRx_{i,j} = CRx_{i,j} U'_{i,j}$

- exists $U$ and $U'$ that $U_j CRx_{i,j} = CRx_{i,j} U'_{i,j}$

Lemma:

- if $U$ acts trivial on $i$, then $U C = C U'$ then if $C$ does not have two-qubit gates, then $U'$ acts trivial on $i$.

if $U$ acts non-trivial on $i$ and non-trivial on $j$ and $U CRx{i,j} = CRx_{i,j} U'$, then U' acts non-trivial on $i$. (also for $j$)

Proof:

$I_j \otimes A CRx_{i,j} = CRx_{i,j} B \otimes I_i$

$P_0 \otimes A + P_1 \otimes A R_j = P_0 B \otimes I_i + P_1 B \otimes I_i R_j$

$\begin{cases} P_0 \otimes A = P_0 B \otimes I_i \\
P_1 \otimes A R_j = P_1 B \otimes I_i R_j
\end{cases}$




We use a operator propagation framework.

\begin{itemize}
  \item A local opreation does not move the operator.
  \item A $CR_x$ extends operator to two-qubit but doesn't move it (lemmas)
\end{itemize}

\begin{lemma}
  In a two-qubit Hilbert space $\mathcal{H}_A \otimes \mathcal{H}_B$ if $V^{(a)} U^{(a,b)} = U^{(a,b)} V'^{(b)}$ for any $V$, then $U$ has two $CR_x$ gates.
\end{lemma}

\begin{lemma}
  In a two-qubit Hilbert space $\mathcal{H}_A \otimes \mathcal{H}_B$ if $V^{(a)} U^{(a,b)} = U^{(a,b)} V'^{(b)}$ and $V^{(b)} U^{(a,b)} = U^{(a,b)} V'^{(a)}$ each for any $V$, then $U$ has three $CR_x$ gates.
\end{lemma}



TODO: some descriptions and text 

\begin{lemma}
  Any two-qubit gate with Schmidt number $2$ could be written as $L_1 \otimes L_2 CR_x(\theta) L_3 \otimes L_4$ where $L_i$ are local unitaries and $CR_x(\theta)$ is a controlled $R_x$ gate.
  \label{lem:decomposition-schmidt-2}
\end{lemma}
\begin{proof}
  Any two-qubit gate with Schmidt number $2$ could be written as $L_1 \otimes L_2 e^{i\alpha ZZ} L_3 \otimes L_4$. 
  Ignoring the local operations, by applying $R_z(-\theta)$ on the first and the second qubit, we will have 
  \begin{equation}
    R_z(-\theta) \otimes R_z(-\theta) e^{i\alpha ZZ} = CR_z(3\theta)
  \end{equation}
  
  that could be converted to $CR_x(\theta)$ by applying $H$ on the second qubit.
\end{proof}

\begin{theorem}[Optimal bridging For $CR_x$]
  Bridging $CR_x$ over $n$ qubits needs at least $4n$ CNOTs.
  \label{thm:bridging-crx}
\end{theorem}
\begin{proof}
  TODO
\end{proof}

The circuit below, shows how to bridge any two-qubit gate with Schmidt number $2$ over $n$ qubits using $4n$ CNOTs and a $CR_x$.
TODO: the shape (similar to CNOT)

\subsection{Previous Version}

\begin{theorem}[X-shaped bridge]
  Any X-shaped bridge, needs at least $4n$ CNOTs.
\end{theorem}

Proof: As we have already defined the bridge gate for CNOT, we can extend it to any two-qubit gate. Therefore a bridge gate is a circuit consisting of two-qubit gates that acts like $U \in \mathcal{H}^{\otimes 2}$ on the first and last qubits, and the identity on the rest of the qubits.

Note that from a simple argument of light cone, we can conclude that the bridge gate must have two chains of $(1, 2), \dots ,(n-1, n)$ and $(n-1, n), \dots, (1, 2)$ gates.

This two chain may intercept (by the definition below) at any qubit between $1$ and $n$, making one of these shapes

\begin{figure}[h]
  \label{fig:intersections}
  \centering
  a)
\begin{quantikz}
  \qw & \qw & \gate[wires=2]{B} & \qw & \qw \\
  \qw & \gate[wires=2]{A} & \qw & \gate[wires=2]{C} & \qw \\
  \qw & \qw & \qw & \qw & \qw
\end{quantikz}
  b)
\begin{quantikz}
  \qw & \gate[wires=2]{A} & \qw & \gate[wires=2]{C} & \qw \\
  \qw & \qw & \gate[wires=2]{B} & \qw & \qw \\
  \qw & \qw & \qw & \qw & \qw
\end{quantikz}
\caption{Possible intersections of two chains of the bridge gate}
\end{figure}

We stick to the case b, and we know that they are equivalent upto changing numbers.

We know that any two-qubit gate (such as $B$) could be decomposed into

\begin{equation}
  B = (L_2 \otimes L_3) (\alpha^{(1)} I_2 I_3 + \alpha^{(2)} X_2 X_3 + \alpha^{(3)} Y_2 Y_3 + \alpha^{(4)} Z_2 Z_3) (L_2' \otimes L_3')
\end{equation}

where $L_i$ and $L_i'$ are local unitaries on the $i$th qubit.

Then, the whole circuit $ABC$ will be equal to 

\begin{equation}
  ABC = \alpha^{(1)} AL_2 (I_1 I_2) L_2' C \otimes L_3 I_3 L'_3 + \alpha^{(2)} AL_2 (I_1 X_2) L_2' C \otimes L_3 X_3 L'_3 + \dots
\end{equation}

And our final goal is to make a bridge that acts like this for all $U_2$s

\begin{equation}
  U_2 ABC = ABC U_2
\end{equation}

By applying $P^{(i)}_3 L_3^\dagger$ and ${L_3'}^\dagger$ from left and right and then applying $\Tr_3$

\begin{equation}
  U_2 \alpha^{(i)} AL_2 (I_1 \otimes P^{(i)}) L_2' C = \alpha^{(i)} AL_2 (I_1 \otimes  P^{(i)}) L_2' C U_2
\end{equation}

Which directly implies

\begin{equation}
  AL_2 (I_1 \otimes P^{(i)}) L_2' C = V^{(i)} \otimes I
\end{equation}

Note that for all $j, k \in \{1, 2, 3\}$
\begin{equation}
  V^{(j)} V^{(0)\dagger} V_{(k)} V^{(0)\dagger} = AL_2 (I_1 \otimes P^{(i)} P^{(j)}) L_2^\dagger A^\dagger = \sum_k i\epsilon_{jkl} V^{(l)}V^{(0)\dagger}
\end{equation}

Here we use a lemma,
\begin{lemma}[Rotated Paulis]
  If $V^{(j)}$s are unitaries that
  \begin{equation}
    V^{(j)} V^{(0)^\dagger} V^{(k)} V^{(0)^\dagger} = \delta_{jk}I + i\epsilon_{jkl} V^{(l)} V^{(0)^\dagger}
  \end{equation}
  then 
  \begin{equation}
    V^{(j)} = U P^{(j)} U^\dagger V^{(0)^\dagger}
  \end{equation}
  that $U$ is a unitary operation and $P^{(j)}$s are Pauli matrices.
\end{lemma}
\begin{proof}
  If we define $Q^{(j)} = V^{(j)} V^{(0)^\dagger}$, then we have shown that $Q^{(j)}$s are hermitian unitaries that $Q^{(0)} = I$ and $Q^{(j)}Q^{(k)} = \delta_{jk}I + i\epsilon_{jkl} Q^{(k)}$, then $Q^{(j)} = U P^{(j)} U^\dagger$ for some $U$.
  Therefore, $V^{(j)} = U P^{(j)} U^\dagger V^{(0)^\dagger}$
\end{proof}


So far we have shown that
\begin{equation}
  AL_2 (I_1 \otimes P^{(i)}) L_2' C = U P^{(i)} U^\dagger V^{(0)^\dagger} \otimes I
\end{equation}

Then, by defining 
\begin{equation}
  \begin{cases} S_{12} = U^\dagger AL_2 \\
    S'_{12} = L_2' C V^{(0)} U
  \end{cases}
\end{equation}

we can say that
\begin{equation}
  S_{12} (I_1 \otimes P^{(i)}) S'_{12} = P^{(i)} \otimes I
\end{equation}


Now, if we define swap-pair set

\begin{definition}[Swap-Pair Set]
  swap-pair set for any unitary operation $U$ is defined as
  \begin{equation}
    \mathrm{SPS}(U) = \{ (X, Y) | X(I \otimes U)Y = U \otimes I \}
  \end{equation}
\end{definition}


\begin{lemma}[Swap-Pair Set for $Z$]
  Any pair that belongs to $\mathrm{SPS}(Z) \cap \mathrm{SPS}(I)$ 
  can be written as $(\mathrm{SWAP} ~ W ~ \mathrm{CV}, \text{h.c.})$ for some single-qubit gate $W$ and a controlled-$V$ gate.
\end{lemma}
\begin{proof}
Note that it is easy to show that $\mathrm{SPS}(I) = \{ (S, S^\dagger) | \forall S \}$ and 
Also, we can show that if $(S, S^\dagger) \in \mathrm{SPS}(Z)$

\begin{equation}
  \begin{aligned}
  &S(I \otimes Z) = (Z \otimes I) S \\
  \rightarrow &\begin{cases}
    S\ket{00} = (Z \otimes I) S\ket{00} \\
    - S\ket{01} = (Z \otimes I) S\ket{01} \\
    S\ket{10} = (Z \otimes I) S\ket{10} \\
    - S\ket{11} = (Z \otimes I) S\ket{11}
  \end{cases} \\
  \rightarrow &\begin{cases}
    S\ket{00} = \ket{0}\ket{\psi_{00}} \\
    S\ket{01} = \ket{1}\ket{\psi_{01}} \\
    S\ket{10} = \ket{0}\ket{\psi_{10}} \\
    S\ket{11} = \ket{1}\ket{\psi_{11}}
  \end{cases} \\
  \end{aligned}
\end{equation}

Noting that $\braket{\psi_{00}}{\psi_{10}}$ and $\braket{\psi_{01}}{\psi_{11}}$ are zero, there are these two maps then

\begin{equation}
  \begin{cases}
  W\ket{\psi_{00}} = \ket{0} \\
  W\ket{\psi_{10}} = \ket{1} \\
  WV\ket{\psi_{01}} = \ket{0} \\
  WV\ket{\psi_{11}} = \ket{1}
  \end{cases}
\end{equation}

Finally, 
\begin{equation}
  S = \mathrm{SWAP}_{1,2} ~ W ~ \mathrm{CV}_{1\to 2}
\end{equation}
\end{proof}

Now, we need to use the lemma to show that $S_{12}$ and $S'_{12}$ are made of $2$ or $3$ CNOTs.
\begin{equation}
  (S, S') \in \mathrm{SPS}(P_i) \forall i
\end{equation}
  
which means that
\begin{equation}
  \begin{cases} U^\dagger AL_2 = \mathrm{SWAP}_{1,2} ~ W ~ \mathrm{CV}_{1\to 2} \\
    L_2' C V^{(0)} U = \mathrm{CV^\dagger}_{1\to 2} ~ W^\dagger ~ \mathrm{SWAP}_{1,2}
  \end{cases}
\end{equation}

which clearly shows that $A$ and $C$ are made of $2$ or $3$ CNOT and have rank $4$.

By induction, we can expand this argument to all of the gates in the circuit, showing th lower bound of $4n$ CNOTs any bridging.


