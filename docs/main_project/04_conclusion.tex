
\chapter{Conclusion}\label{chap:conclusion}

\section{Future works}

% TODO the ancilla case
As like as many other efforts, we assumed that any qubit in the device is used and is continuously carrying information. This might not be a bad idea in NISQ devices, where any single qubit is precious and the number of qubits is limited. However, because of its interesting results and its potential to be used in future devices, we will briefly discuss the case where we have ancilla qubits as well.

Bridging class I gates over $n$ ancilla qubit needs $2n + O(1)$ CNOTs and this number for class II gates is $4n + O(1)$, which shows a significant improvement in comparison to th previous results.
{ \color{red} TODO: show circuits for these cases 
  mention network coding as well \cite{ho2008}
}
