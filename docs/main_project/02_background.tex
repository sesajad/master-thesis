\chapter{Background}\label{chap:background}

We should be concerned with two key findings from this chapter. First, the methodology employed to deconstruct the problem of quantum compilation, along with its underlying assumptions. Secondly, the facts and techniques that will also be used for our research. Therefore, here we try to review both of these aspects to draw a big picture of our notion of quantum compilation and also to review the existing techniques related to our approach and we do these two alongside each other.

\section{Quantum Compilation}
The term ``quantum compilation'' can refer to any process that transforms a higher-level description of a quantum algorithm into a lower-level description~\cite{hundt2022}. In the majority of works \cite{zulehner2018,childs,cross2022,sivarajah2021,qiskit2023,paler2021}, circuits are the description used, and thereby the compilation is done by transforming a general quantum circuit into a circuit that is compatible with specific hardware. Given that the problem of finding the most optimal circuit (with respect to a sense of complexity such as depth or number of gates) has been proven to be NP-hard \cite{siraichi2018}, the research primarily centres on deconstructing this problem into smaller components and devising techniques to effectively balance between the agility of the process and the quality of the solution. This mirrors the approach adopted in classical compilation \cite{allen2001}.

However, the clear structure for this breakdown has not been firmly established, with numerous diverse approaches in play. As such, our goal is to provide a comprehensive overview of the issue and identify common patterns in the existing literature. This overall picture will then serve as a reference point throughout the rest of this thesis. To achieve this overview, it is crucial to define \textbf{circuit transformation} as a process that preserves the essential meaning of the circuit, ensuring that the circuit's output remains consistent for any input before and after the transformation. Yet, this process alters the circuit to adhere to specific constraints or optimises it for particular goals.

Thus, \textbf{quantum compilation} in our essay will be a sequence of circuit transformations where each transformation either enforces a constraint or optimises the circuit (where the optimisation itself can be perceived as a soft constraint). The primary constraints imposed upon the circuit arise from the characteristics of the hardware and are listed below:

\begin{itemize}
  \item \textbf{Gate set}: The set of gates physically available in the hardware.
  \item \textbf{Connectivity}: The connectivity between qubits in the hardware topology.
\end{itemize}
  
Furthermore, optimisations that correspond to various degrees of freedom (e.g., qubit assignments, choosing among equivalent subcircuits) can be applied. They are commonly pursuing these goals:

\begin{itemize}
  \item \textbf{Complexity reduction}: Decrease the number of gates or the depth of the circuit.
  \item \textbf{Preparation for constraints}: Minimise violation of the mentioned constraints before imposing them.
  \item \textbf{Error mitigation}: Reduce the error of the circuit with respect to the hardware, especially in NISQ devices.
\end{itemize}

Quantum compilation frameworks mainly use a similar approach, they introduce transformations that are each is somehow seeking one of the goals defined before, tket even uses the term predicates for the constraints that are similar to these \cite{sivarajah2021}. Typical transformations in their compilation process are, decomposing gates (imposing gate set constraint), assigning qubits (optimisation of connectivity as a soft constraint), routing (imposing connectivity constraint), and further optimisations (reducing complexity). Yet the order of these transformations is not the same everywhere, for example, some \cite{lao2021,qiskit2023} defer the gate synthesis after the routing while some \cite{wille2020,sivarajah2021} do it otherwise.

With this background established, we now dive into details on gate set and connectivity constraints, and circuit optimisation techniques for reducing complexity.

\subsection{Gate synthesis}

Imposing the gate set constraints, which is known as the gate synthesis, is one of the oldest subroutines in quantum compilation. The problem here is to decompose a general $n$-qubit gate into a sequence of gates from the physically available gates on the device. If the set of available gates is able to produce any arbitrary gate, we call it a universal gate set \cite{barenco1995}. Here, while we review the results for the synthesis of one-, two-, and more-than-two-qubit gates, we try to define mathematical objects that will be used in the rest of the thesis. For this purpose, the gate set constraint is defined as below:

\begin{definition}[Gate set constraint]
  The gate set constraint is represented by a set of unitary operators $G$ that are the set of available gates on the device, and we say a circuit (or a sequence of gates) is compatible with the gate set constraint if it is composed of only gates from $G$.

  For our purpose, $G$ is the set of all one-qubit gates plus CNOT gate.
\end{definition}

While most of the devices have the capability to perform arbitrary one-qubit gates, with a negligible cost, there is a famous result for the synthesis of one-qubit gates, called Solovay-Kitaev theorem \cite{dawson2006}, that formalises approximate synthesis of any one-qubit gate using only two distinct one-qubit gates as the gate set.

For two-qubit gates, it is known that the set of all one-qubit gates together with CNOT can produce any two-qubit gate and it can be done by up to three CNOTs (and it is optimal) \cite{vatan2004,vidal2004}. One of the famous gates that needs three CNOTs is SWAP. Using CNOT as the only available two-qubit gate although is common in theoretical studies of compilation \cite{zulehner2018,siraichi2018,li2019,zhang2021,itoko2019,murali2019} there are other continuous gate sets such as $\mathrm{fSim}(\theta, \phi)$ \cite{foxen2020} or other Hamiltonian evolutions \cite{childsa}. Also, the evolution of XYZ Heisenberg interaction Hamiltonian, has been used as an intermediate gate set or a tool for analytical analysis in some works \cite{sousa2006,vidal2004}. The importance of this Hamiltonian is because of a decomposition (Cartan's KAK decomposition, Cirac-Kraus decomposition, or Khaneja-Glaser decomposition) that states any two-qubit gate can be made by only one evolution of $XYZ$ Hamiltonian and four one-qubit gates \cite{kraus2001,khaneja2001}. Here we state this fruitful decomposition as a lemma:

\begin{lemma}[KAK decomposition \cite{tucci2005}]\label{lem:kak_decomposition}
For any $U \in U(4)$, there exist $V_1, V_2, V_3, V_4 \in U(2)$ together with $\alpha, \beta, \gamma \in \mathbb{R}$ such that
\begin{equation}
  U = (V_1 \otimes V_2) e^{i\alpha X\otimes X + i\beta Y\otimes Y + i\gamma Z\otimes Z} (V_3 \otimes V_4).
  \end{equation}
\end{lemma}

For more than two-qubit gates, first TOFOLLI gate was used to prove the universality \cite{barenco1995}, later it was proven that it is not necessary. Although there are a few ways to synthesise a general $n$-qubit gate  \cite{sousa2006,shende2006}, they are not commonly used because of its inherent inefficiency. It is known that it could not be better than $O(4^n)$ gates. \cite{shende2006}

Synthesis of local Hamiltonian evolution, as a special case of $n$-qubit gate, is also studied as it is important for simulation purposes \cite{childs2018}. Most of the research is built upon Suzuki-Trotter decomposition~\cite{trotter1959, suzuki1991}, that approximates the time evolution of a Hamiltonian with a sequence of time evolutions of its terms. While most of the works rely on the first-order Suzuki-Trotter decomposition (aka Lie-Trotter formula) ~\cite{sivarajah2021, qiskit2023}, there are a few efforts to analyse higher-order errors~\cite{childs2021} or to use other decomposition, such as a random sampling decomposition (QDRIFT) ~\cite{campbell2019}. Besides the gate synthesis, Hamiltonian compilation has implications that can be used in other parts of the compilation process, for example ~\cite{lao2021} defines routing and scheduling algorithms specifically for Hamiltonian compilation.

\subsection{Qubit Allocation and Routing}

The implementation of the qubits into a physical substance needs to overcome so many technical difficulties, and it will be impossible to have a fully connected device with a large number of qubits in the near future. Therefore, the connectivity constraint is always a part of the compilation process in the NISQ era. A typical device looks like Figure~\ref{fig:connectivity}, that the nodes are qubits and the edges are showing the pairs that a two-qubit gate can be applied on them.

\begin{figure}[h!]
  \centering
  a) \includegraphics*[width=5cm]{shapes/ibm_dev}
  b) \includegraphics*[width=5cm]{shapes/ibm_conn}
  \caption{a) IBM Eagle processor, a 127 qubit device, b) IBM Eagle connectivity graph \cite{ibm2021}}
  \label{fig:connectivity}
\end{figure}

Connectivity constraint will be similarly defined and then, we will review its literature which has drawn so much attention in the recent years.

\begin{definition}[Connectivity constraint]
  The connectivity constraint is represented by a graph $G = (V, E)$ where $V$ is the set of qubits and $E$ is the set of edges between qubits. We say a circuit (or a sequence of gates) is compatible with the connectivity constraint if there is a map from qubits to vertices of $G$ and for each two-qubit gate between two qubits, there is an edge between the corresponding vertices in $G$.
\end{definition}

Researchers often tackle the problem of resolving connectivity constraint, by defining two subproblems, qubit allocation and routing. The definition of them may vary in the literature, but we can roughly define the \textbf{qubit allocation} qubit allocation we roughly mean a mapping from logical qubits of a quantum circuit to physical qubits of a device in way that minimises the circuit's complexity overhead due to routing. And by \textbf{routing} we mean the problem of finding an optimal circuit that is compatible with device connectivity and is equivalent to the circuit. Results suggest that the qubit allocation could affect the complexity of the circuit by $10\%$ in realistic scenarios~\cite{paler2019}. This fact justifies such a break-down of connectivity constraint into two subproblems furthermore.

Qubit allocation shares common traits with other resource allocation problems \cite{alicherry2012},\cite[pp. 440-444]{allen2001} and it is not surprising that it is NP-hard for arbitrary connectivity graphs, this can be shown easily by a reduction from graph isomorphism~\cite{siraichi2018}. Note that real-world devices are not arbitrary graphs and it might be exactly solvable in reasonable time for certain families of graphs.

There have been attempts to find the optimal solution by searching for arbitrary connectivity graphs, which can be feasible for small devices~\cite{siraichi2018}. But, the most common approach in the research is to use a heuristic~\cite{zhang2021, itoko2019, cowtan2019,paler2019, murali2019} together with a search algorithm (such as BFS, $A^*$\cite{zulehner2018}, or others\cite{li2019}) to find a reasonable solution. Some of these efforts have also been implemented in current quantum compilers \cite{qiskit2023,sivarajah2021}.

Treating routing problem, could be done similarly by thinking of routing as a subsequent qubit reallocations. By introducing a search space of partial allocations (that are partial functions), we can use a unified search space for both qubit allocation and routing~\cite{zulehner2018,childs}. While this will unlock the possibility of using the same algorithm for both qubit allocation and routing that is used in \cite{zulehner2018}, the most of the recent works prefer to treat them separately~\cite{li2019,lao2021,childs}.

Using heuristic algorithms for routing is a common practice in the literature~\cite{zulehner2018, itoko2019, cowtan2019, li2019}, while there are efforts to find the an exact algorithm \cite{itoko2019} that will be inefficient for general case, and by imposing some restrictions on the connectivity graph, we can solve the problem in polynomial time~\cite{childs}. For example, the problem is solvable in polynomial time for path graphs, complete graphs, tree graphs, and product graphs. These results have been built upon classical problems that are already known like token swapping and routing via matching~\cite{banerjee2017}.

While the SWAP gate plays a crucial role in most of the routing algorithms, there are a few efforts that go beyond the SWAP gates, like using bridged CNOT (sometimes called remote CNOT) gate over 1 qubit \cite{itoko2019,siraichi2018,sivarajah2021}. By bridged CNOT gate, we mean any sequence of gates that acts as CNOT between two non-adjacent qubits without explicitly having a SWAP gate in it. An example of such gate is depicted in Figure~\ref{fig:bridged-cnot-1}a and it can be compared with the case that SWAPs are explicitly used, which is shown in Figure~\ref{fig:bridged-cnot-1}b.

\begin{figure}[h!]
  \label{fig:bridged-cnot-1}
  \centering
  \begin{tikzpicture}
    \node at (-4.3, 0.8) {a)};
    \node[scale=0.7] at (0,0) {
    \begin{quantikz}
    \qw & \ctrl{2} & \qw \midstick[3,brackets=none]{=} & \qw & \ctrl{1} & \qw & \ctrl{1} & \qw \\
    \qw & \qw & \qw & \ctrl{1} & \targ{} & \ctrl{1}  & \targ{} & \qw \\
    \qw & \targ{} & \qw & \targ{} & \qw  & \targ & \qw  & \qw &  \qw \\
  \end{quantikz}
  };
  \node at (-4.3, -1.2) {b)};
  \node[scale=0.7] at (1,-2) {
    \begin{quantikz}
      \qw & \ctrl{2} & \qw \midstick[3,brackets=none]{=} & \swap{1} & \qw & \swap{1} & \qw \midstick[3,brackets=none]{=} & \ctrl{1} & \targ{} & \ctrl{1} & \qw &\ctrl{1} & \targ{} & \ctrl{1} & \qw \\
      \qw & \qw & \qw & \swap{} & \ctrl{1} & \swap{} & \qw & \targ{} & \ctrl{-1}& \targ{} & \ctrl{1} & \targ{} & \ctrl{-1}& \targ{} & \qw \\
      \qw & \targ{} & \qw  & \qw & \targ{} & \qw & \qw & \qw & \qw & \qw & \targ & \qw & \qw & \qw & \qw  & \qw \\
    \end{quantikz}
    };
    \end{tikzpicture}
  \caption{a) A bridged CNOT gate over one qubit, b) The baseline implementation of bridged CNOT gate over one qubit}
\end{figure}

\begin{figure}[h!]
  \label{fig:bridged-swap-1}
  \centering
  \begin{tikzpicture}
    \node at (-4.3, 0.8) {a)};
    \node[scale=0.7] at (0,0) {
    \begin{quantikz}
      \qw & \swap{2} & \qw \midstick[3,brackets=none]{=} & \ctrl{1} & \qw &\targ{}& \qw & \ctrl{1}& \qw &\targ & \qw & \qw & \qw \\
      \qw & \qw & \qw & \targ{} & \targ{} & \ctrl{-1} & \ctrl{1} & \targ{} & \targ{} & \ctrl{-1} & \ctrl{1} & \qw \\
      \qw & \swap{} & \qw & \qw & \ctrl{-1} & \qw &\targ{}& \qw & \ctrl{-1}& \qw &\targ & \qw & \qw \\
  \end{quantikz}
  };
  \node at (-4.3, -1.2) {b)};
  \node[scale=0.7] at (1,-2) {
    \begin{quantikz}
      \qw & \swap{2} & \qw \midstick[3,brackets=none]{=} &
      \swap{1} & \qw & \swap{1} & \qw \midstick[3,brackets=none]{=} &
      \ctrl{1} & \targ{} & \ctrl{1} & \qw & \qw & \qw &\ctrl{1} & \targ{} & \ctrl{1} & \qw \\
      \qw & \qw & \qw &
       \swap{} & \swap{1} & \swap{} & \qw & 
      \targ{} & \ctrl{-1}& \targ{} & \ctrl{1} & \targ{} & \ctrl{1} & \targ{} & \ctrl{-1}& \targ{} & \qw \\
      \qw & \swap{} & \qw  &
      \qw & \swap{} & \qw & \qw &
      \qw & \qw & \qw & \targ{} & \ctrl{-1} & \targ{} & \qw & \qw & \qw  & \qw \\
    \end{quantikz}
    };
    \end{tikzpicture}
  \caption{a) A bridged SWAP gate over one qubit, b) The baseline implementation of bridged SWAP gate over one qubit}
\end{figure}

This idea could be broadened to any two-qubit gate, even SWAP itself, (see Figure~\ref{fig:bridged-swap-1}). This is the problem people have been working on under different names \cite{shende2006} and, here we define bridged gates for an arbitrary connectivity constraint in the definition below:

\begin{definition}[Bridged gate]
  Given a connectivity constraint $G = (V, E)$, a target two-qubit gate called $T$ defined on two qubits $a, b \in V$ then, a \textbf{bridged} $T$ gate is a sequence of gates that obeys the connectivity constraint and is identical to $T$.
\end{definition}

For the sake of readability, we may omit specifying the connectivity constraint when it is a linear connectivity constraint. From this definition, we can see that the baseline implementation of a bridged gate can be done using SWAPs (like Figure~\ref{fig:bridged-cnot-1}b and Figure~\ref{fig:bridged-swap-1}b), however we are looking for more efficient implementations that are not necessarily trivial (like Figure~\ref{fig:bridged-cnot-1}a and Figure~\ref{fig:bridged-swap-1}a). The baseline implementation is also shown in the following corollary.

\begin{corollary}[Baseline implementation of a bridged gate]\label{cor:baseline-bridged}
  For any two-qubit gate $T$, A bridged $T$ gate defined on qubits $1, n \in V$ where every $(i, i + 1) \in E$, can be implemented for an even $n$ using the following sequence of gates:
  \begin{equation}
    T^{1,n} = \prod_{i=n/2-1}^{1} \SWAP^{i,i+1} \otimes \SWAP^{n-i+1, n-i} T^{n/2, n/2+1} \prod_{i=1}^{n/2-1} \SWAP^{i,i+1} \otimes \SWAP^{n-i+1, n-i},
  \end{equation}
  and for an odd $n$ we can respectively use:
  \begin{equation}
    T^{1,n} = \SWAP^{n,n-1} T^{1,n-1} \SWAP^{n,n-1}.
  \end{equation}

  Where $\SWAP^{i,j}$ is a SWAP gate between qubits $i$ and $j$, $T^{i,j}$ is a $T$ gate applied on $i$ and $j$, and note that $\prod_{i=a}^b A_i = A_b A_{b-1} \dots A_a$.
\end{corollary}

In addition to those efforts, there are tries to totally avoid the SWAP gates and directly apply the routing process by transforming the circuit into a compatible circuit using CNOTs and $R_z$s \cite{nash2020,kissinger2019}. Note that this process is not easily expandable to arbitrary gates and circuits, and because of its nature (of working directly with a representation of the whole circuit), its scalability is questionable. Despite that, an important result of this approach is to generate bridged CNOT gates over more than one qubits in a systematic way \cite{nash2020}. The gate that can be generated is written in Equation \ref{eq:bridged-cnot-m-n} and is shown in Figure \ref{fig:bridged-cnot-m-n}. This generalisation uses $4n + O(1)$ CNOTs and has a depth of $4n + O(1)$ for a bridged CNOT over $n$ qubits, which is not generally better in comparison to the baseline implementation that uses $6n + O(1)$ CNOTs, with a depth of $3n + O(1)$.

\begin{equation}
  \label{eq:bridged-cnot-m-n}
  \CNOT^{1\to n} = \qty[\prod_{i=n - 2}^{2}\CNOT^{i + 1\to i} \prod_{i=1}^{n - 1}\CNOT^{i + 1\to i}]^2,
\end{equation}
where $\CNOT^{i\to j}$ is a CNOT gate with control on qubit $i$ and target on qubit $j$.

\begin{figure}[h!]
  \label{fig:bridged-cnot-m-n}
  \centering
  \begin{tikzpicture}
    \node[scale=0.7] {
\begin{quantikz}
\qw &\ctrl{5}&\qw\midstick[6,brackets=none]{=}&
 \qw    &\qw     &\qw     &\qw     &\ctrl{1}& \qw    & \qw    & \qw     &
 \qw    &\qw     &\qw     &\qw     &\ctrl{1}& \qw    & \qw    & \qw     & \\
\qw & \qw    & \qw    &
 \qw    &\qw     &\qw     &\ctrl{1}& \targ{}&\ctrl{1}& \qw    & \qw     &
 \qw    &\qw     &\qw     &\ctrl{1}& \targ{}&\ctrl{1}& \qw    & \qw     & \\
\qw & \qw    & \qw    &
 \qw    &\qw     &\ctrl{1}& \targ{}& \qw    &\targ{} &\ctrl{1}& \qw     &
 \qw    &\qw     &\ctrl{1}& \targ{}& \qw    &\targ{} &\ctrl{1}& \qw     & \\
\qw & \qw    & \qw    &
 \qw    &\ctrl{1}&\targ{} & \qw    & \qw    & \qw    &\targ{} &\ctrl{1} &
 \qw    &\ctrl{1}&\targ{} & \qw    & \qw    & \qw    &\targ{} &\ctrl{1} & \\
\qw & \qw    & \qw    &
\ctrl{1}&\targ{} &\qw     & \qw    & \qw    & \qw    & \qw    &\targ{}  &
\ctrl{1}&\targ{} &\qw     & \qw    & \qw    & \qw    & \qw    &\targ{}  & \\
\qw &\targ{} & \qw    &
\targ{} & \qw    & \qw    & \qw    & \qw    & \qw    & \qw     & \qw    &
\targ{} & \qw    & \qw    & \qw    & \qw    & \qw    & \qw     & \qw    & \\
\end{quantikz}
};
\end{tikzpicture}
  \caption{A bridged CNOT gate over 6 qubits}
\end{figure}

\subsection{Optimisations}

Circuit optimisation is the set of techniques to transform the circuit into a more efficient one, without imposing or lifting any constraint. Circuit optimisation is often achieved by applying simplification rules to the circuit~\cite{pointing2021}. These simplification rules are usually based on the commutation relations of the gates~\cite{itoko2019,qiskit2023,sivarajah2021}.

Therefore the optimisation could be seen as a search problem in the space of equivalent circuits that are explored by applying the simplification rules. The matter that to which extent the search space is explored is a degree of freedom that is called optimisation level \cite{qiskit2023,sivarajah2021}, which is shared with classical compilers as well \cite{aho1986}.