\chapter{Discussion}\label{chap:discussion}

\section{Problem Statement}

We have already reviewed the main research themes around quantum compilation in Chapter~\ref{chap:background}. We have seen that one of the main challenges is in imposing connectivity constraints on the circuit. Furthermore, because of the huge dependence of most of the research on swapping qubits, putting together with the fact that swapping is the most expensive two-qubit gate, it is important to study the alternatives for swapping. In this chapter, we introduce our research problem based on the definition of bridging that we discussed earlier.

While our aim in the most broad term is to find alternatives to SWAP gate, to apply gates with respect to connectivity constraints, we start with a more specific problem. We seek for optimal bridged gate, and by optimal we mean both the least number of CNOTs and the least depth. Note that, this problem is going to be studied for an arbitrary connectivity constraint $G$.

The relation of this problem to the approach of \cite{nash2020,kissinger2019} is that we are looking for bridged gates that can be adopted in the any compilation process, in contrast to theirs that is defining a whole new compilation process. In terms of the results, while some of our preliminary results such as a simple bridged CNOT gate can be derived from their techniques too, however, the main contribution of our work is to show the optimal bridged gate for any two-qubit gate, while those efforts are limited to compiling the gates that are combination of CNOTs and $R_z$s.

\section{Two-qubit Gate Classes}

Our discussion heavily relies on the KAK decomposition of two-qubit gates. We have already introduced it in Lemma~\ref{lem:kak_decomposition}. Here we introduce a few classes of two-qubit gates, based on their KAK decomposition.

\begin{definition}[Two-qubit Gate Classes]
  Based on the KAK decomposition of two-qubit gates, that is 
  \begin{equation}
    U = (V_1 \otimes V_2) e^{i \alpha X \otimes X + i \beta Y \otimes Y + i \gamma Z \otimes Z} (V_3 \otimes V_4),
  \end{equation}
  we define the following classes of two-qubit gates:
  \begin{itemize}
    \item \textbf{Class nil}: Two-qubit gates that $\alpha = \beta = \gamma = 0$.
    \item \textbf{Class I:} Two-qubit gates that only one of $\alpha, \beta, \gamma \neq 0$.
    \item \textbf{Class II:} Two-qubit gates that at least two of $\alpha, \beta, \gamma \neq 0$.
  \end{itemize}
\end{definition}

The gates in class nil are local gates that are applied on each qubit separately, so they are irrelevant to our discussion about bridging. The gates in class I, such as CNOT, CZ and etc., can be decomposed to one $\mathrm{C}R_x$ gate upto local gates. This fact is shown in Lemma~\ref{lem:crx-decomposition} and enables us to bridge them.

\begin{lemma}\label{lem:crx-decomposition}
  For any two-qubit gate $U$ in class I, it can be decomposed as 
  \begin{equation}
    U = W_1 \otimes W_2 \mathrm{C}R_x(\theta) W_3 \otimes W_4,
  \end{equation}
  where $\mathrm{C}R_x(\theta)$ is a controlled rotation gate around $x$ axis, with control on the first qubit and target on the second qubit.
\end{lemma}
\begin{proof}
  Without lose of generality, we can assume that $\gamma \neq 0$ and $\alpha = \alpha = 0$. Now, we can write
  \begin{equation}
    U = (V_1 \otimes V_2) e^{i \gamma Z \otimes Z} (V_3 \otimes V_4).
  \end{equation}
  Then, similar to the way that we derive $\mathrm{C}Z$ and $\mathrm{CNOT}$ gates, we can write
  \begin{equation}
    \begin{aligned}
    U &= (V_1 \otimes V_2 R_z(-2\gamma)) CR_z(4\theta) (V_3 \otimes V_4), \\
    &= (V_1 \otimes V_2 R_z(-2\gamma) H) CR_x(4\theta) (V_3 \otimes H V_4).
    \end{aligned}
  \end{equation}
  Now we $W_i$s are easily derived from $V_i$s.
\end{proof}

The gates in class II, such as SWAP, cannot be bridged any better than the naive solution with SWAPs. This fact is shown later using the special feature of these gates in Lemma~\ref{lem:no-trivial-commutation}. 
\section{Bridged Gates}

Here we introduce the bridged $T$ gate for any $T$ that belongs to class I. This gate is shown under a linear connectivity constraints, so it can be used for any other connectivity constraint (by using the connecting path between the qubits). An example of this gate (bridging over $4$ qubits) is shown in Figure~\ref{fig:bridged-class-i}. Note that $V_i$s and $\theta$ can be computed using Lemma~\ref{lem:crx-decomposition}

\begin{figure}[ht]
  \label{fig:bridged-class-i}
  \centering
  \begin{tikzpicture}
    \node[scale=0.7] {
\begin{quantikz}[transparent]
  \qw &\gate[6]{T}                        &\qw\midstick[6,brackets=none]{=}&
  \qw &\gate{V_3} & \ctrl{5} & \gate{V_1} &\qw\midstick[6,brackets=none]{=}&
  \gate{V_3} & \targ{}  & \qw     &\ctrl{1}& \qw    & \qw    & \qw    &\ctrl{1}& \qw     &\targ{}&\gate{V_1}\\
  \qw &\linethrough &\qw &
  \qw &\qw &\qw &\qw &\qw &
  \qw & \ctrl{-1}&\targ{}  & \targ{}&\ctrl{1}& \qw    &\ctrl{1}&\targ{} &\targ{}  &\ctrl{-1}&\qw\\
  \qw &\linethrough &\qw &
  \qw &\qw &\qw &\qw &\qw &
  \qw & \qw &\ctrl{-1}& \qw    & \targ{}&\ctrl{1}&\targ{} & \qw    &\ctrl{-1}&\qw & \qw \\
  \qw &\linethrough &\qw &
  \qw &\qw &\qw &\qw &\qw &
  \qw & \qw &\targ{}  & \qw    &\ctrl{1}& \gate{R_x(\theta)}&\ctrl{1}& \qw    &\targ{}  &\qw & \qw\\
  \qw &\linethrough &\qw &
  \qw &\qw &\qw &\qw &\qw &
  \qw & \targ{}  &\ctrl{-1}&\ctrl{1}& \targ{}& \qw    &\targ{} &\ctrl{1}&\ctrl{-1}&\targ{}&\qw \\
  \qw &  & \qw &
  \qw &\gate{V_4} & \gate{R_x(\theta)} & \gate{V_2} & \qw&
  \gate{V_4} & \ctrl{-1}& \qw     & \targ{}& \qw    & \qw    & \qw    &\targ{} & \qw     &\ctrl{-1}& \gate{V_2} 
  \end{quantikz}
  };
  \end{tikzpicture}
  \caption{A bridged class I gate over 6 qubits}
\end{figure}
  
This circuit above can be mathematically defined as follows:

\begin{theorem}[Bridged Class I Gate]
  Any class I gate $T$ with decomposition of Lemma~\ref{lem:crx-decomposition}, acting on qubits $1$ and $n$ could be bridged over $n-2$ qubits with linear connectivity using the following circuit  
  \begin{equation}
    T = (V_1 \otimes V_2) B^\dagger_n \mathrm{C}R_x^{\lceil n/2 \rceil \to \lceil n/2\rceil+1}(\theta) B_n (V_3 \otimes V_4),
  \end{equation}
  where $B_n$ for $n = 4$ is defined as follows:
  \begin{equation}
    B_4 = (\CNOT^{1 \to 2} \otimes \CNOT^{3 \to 4}) (\CNOT^{2 \to 1} \otimes \CNOT^{4 \to 3}),
  \end{equation}
  and for an even $n \ge 6$ could be written as follows:
  \begin{equation}
    \begin{aligned}
    B_{n} = &(\CNOT^{n/2 - 1 \to n/2} \otimes \CNOT^{n/2 + 1 \to n/2 + 2}) \\
    &(\CNOT^{n/2 - 2 \to n/2 - 1} \otimes \CNOT^{n/2 + 2 \to n/2 + 3}) \\
    &\prod_{i=1}^{n/2 - 3} (\CNOT^{i \to i+1} \otimes \CNOT^{i+3 \to i+2} \otimes \CNOT^{n-i-1 \to n-i-2} \otimes \CNOT^{n-i \to n-i+1}) \\
    &(\CNOT^{3 \to 2} \otimes \CNOT^{n-1 \to n-2}) \\
    &(\CNOT^{2 \to 1} \otimes \CNOT^{n \to n-1}).
    \end{aligned}
  \end{equation}
  Then, $B_n$ for any odd $n$ could be recovered by removing $n$th qubit and its corresponding CNOTs from $B_n$.
\end{theorem}
\begin{proof}
  In order to prove the correctness of this circuit, we derive it from the naive solution with SWAPs to prove that these two are equivalent. The first step to decompose the gate by $\mathrm{C}R_x$ and to push local gates out of the bridging process. Then, by decomposing each SWAP gate to three CNOTs, like below
\begin{equation}
  \mathrm{SWAP} = \CNOT^{2 \to 1} \CNOT^{1 \to 2} \CNOT^{2 \to 1},
\end{equation}
and using these commutation relations
\begin{equation}
  \begin{cases}
  [\mathrm{C}R_x^{i \to i-1}(a), \mathrm{C}R_x^{i \to i+1}(b)] = 0 \\
  [\mathrm{C}R_x^{i-1 \to i}(a), \mathrm{C}R_x^{i+1 \to i}(b)] = 0
  \end{cases}
\end{equation}
we can simpilify one out of three CNOTs in a manner that is shown in Figure~\ref{fig:bridged-class-i-proof}. Further simplifications these commutation relations reduces the depth from $2n + O(1)$ to $n + O(1)$ by replacing CNOTs to improve parallelization.

\begin{figure}[ht]
  \label{fig:bridged-class-i-proof}
  \centering
  \begin{tikzpicture}
    \node[scale=0.7] {
      \begin{quantikz}
        \qw & \targ{} & \ctrl{1} & \targ{} & \qw & \targ{} & \ctrl{1} & \targ{} & \qw\midstick[5,brackets=none]{=} & 
        \qw & \targ{} & \ctrl{1} & \qw & \ctrl{1} & \targ{} & \qw \\
        \qw & \ctrl{-1} & \targ{} & \ctrl{-1} & \ctrl{3} & \ctrl{-1} & \targ{} & \ctrl{-1} & \qw &
        \qw & \ctrl{-1} & \targ{} &\ctrl{3} & \targ{} & \ctrl{-1} & \qw \\
        \qw & \qw & \qw & \qw & \qw & \qw & \qw & \qw & \qw &
        \qw & \qw & \qw & \qw & \qw & \qw & \qw  \\
        \qw & \qw & \qw & \qw & \qw & \qw & \qw & \qw & \qw &
        \qw & \qw & \qw & \qw & \qw & \qw & \qw \\
        \qw & \qw & \qw & \qw & \gate{\mathrm{C}R_x(\theta)} & \qw & \qw & \qw & \qw &
        \qw & \qw & \qw &\gate{\mathrm{C}R_x(\theta)} &\qw & \qw & \qw \\
      \end{quantikz}
    };
  \end{tikzpicture}
  \caption{Simplifying one CNOT in a SWAP gate}
\end{figure}
\end{proof}

The formula above shows that the circuit uses $4n + O(1)$ CNOTs with the depth $n + O(1)$ (noting that $\mathrm{C}R_x$ needs two CNOTs \cite[chapter 4]{nielsen2010}). While we know that the easiest way to bridge a target gate $T$ over $n$ qubits with linear connectivity is to use consequtive SWAPs, which will need $2n + O(1)$ SWAPs ($6n + O(1)$ CNOTs) and has at least $3n + O(1)$ depth. Moreover, for bridging CNOTs, this circuit has approximately the same number of CNOTs as the bridged CNOT gate in Equation~\ref{eq:bridged-cnot-m-n} and Figure~\ref{fig:bridged-cnot-m-n}, however, its depth is four times less than that.

% TODO, maybe add the bridged swap gate over 2 here, to have some fun!

\section{Proof of Optimality}

These proofs are based on the idea that if two gate sequence $A = A_n \dots A_2 A_1$ and $B = B_m \dots B_2 B_1$ are equal, if we multiply an arbitrary gate $U_{0}$ from right
\begin{equation}
  A_n \dots A_2 A_1 U = B_m \dots B_2 B_1 U,
\end{equation}
then, if we move $U_{0}$ to the left, in a process that $U_0$ is converted to $U_{a1}$ like $U_{a1} A_1 = A_1 U_{0}$ and similarly for $B$s, so on, we will finally have
\begin{equation}
  U_{an} A_n \dots A_2 A_1 = U_{bm} B_m \dots B_2 B_1.
\end{equation}

Because of the assumption of equality, we can conclude that $U_{an} = U_{bm}$. This means that the necessary condition for equality of $A$ and $B$ is $U_{an} = U_{bm}$. In order to be able to phrase this condition in a more useful way, we define the following concept:

\begin{definition}[Successor]
  If $AB = BC$, $C$ is called the successor of $A$ under $B$.
\end{definition}

With this definition, the whole argument could be summarized as the necessary condition for two gate sequences to be equal is that the successor of any arbitrary gate under these two gates should be equal.

\begin{definition}[Trivial/Nontrivial Action]
  A gate $U$ defined on a set of qubits $Q$ is acting trivial on qubit $q \in Q$ if $U$ can be written as 
  \begin{equation}
    U = U' \otimes I_q
  \end{equation}
  where $U'$ is a gate defined on $Q \setminus \{ q \}$ and $I_q$ is the identity gate on qubit $q$.
  Respectively, $U$ is acting nontrivial on $q$ if it cannot be written in the above form.
\end{definition}

Now, we can start proving the optimality of the bridged $T$ gate for different classes, starting by class I.

\subsection{Class I}

The sketch of this proof consists of first reducing the problem to bridging a $\mathrm{C}R_x^{1\to n}$ gate. Then because of the nontrivial (on $1$ and $n$) successors of $X_1$ and $Z_n$ under $\mathrm{C}R_x^{1\to n}$, we can conclude that any sequence of gates that is bridged $\mathrm{C}R_x^{1\to n}$ needs to have a minimum number of CNOTs, and a minimum depth, to produce such nontrivial successors. These two bounds are achieved by the bridged $T$ gate.

So, to start the proof, we first establish two facts about the successors:
\begin{corollary}[No-go for One-qubit Gates]\label{cor:no-go-one-qubit}
  For any gate acting trivial on $t$, its successor under any one-qubit gate on $t$ is still trivial on $t$. 
\end{corollary}

\begin{lemma}[No-Move for Class I]\label{lem:no-move-class-i}
  For any gate $X$ acting nontrivial on $n$ and trivial on $t$, its successor $Y$ under any class I gate $U$ acting on $n$ and $t$ , is nontrivial on $n$.
\end{lemma}
\begin{proof}
  Noting that $X$ must be trivial on $t$ and nontrivial on $n$, $X$ could be decomposed as $X' \otimes I_t$.
  Using the definition of successor in form of $U^\dagger X U = Y$, we can write

  \begin{equation}
    \begin{aligned}
    Y &= U^\dagger (X' \otimes I_t) U  \\
    &= (V_3^\dagger \otimes V_4^\dagger)\mathrm{C}R_x(-\theta) (V_1^\dagger X' V_1 \otimes I) \mathrm{C}R_x(\theta) (V_3 \otimes V_4) \\
    &= (V_3^\dagger \otimes V_4^\dagger)(\dyad{0} + R_x(-\theta) \otimes \dyad{1}) (V_1^\dagger X' V_1 \otimes I) (\dyad{0} + R_x(\theta) \otimes \dyad{1}) (V_3 \otimes V_4) \\
    &= (V_3^\dagger \otimes V_4^\dagger)(V_1^\dagger X' V_1 \dyad{0} + R_x(-\theta) V_1^\dagger X' V_1 R_x(\theta) \dyad{1}) (V_3 \otimes V_4)
    \end{aligned}
  \end{equation}

  We have used the decomposition of Lemma~\ref{lem:crx-decomposition} for $U$ and expanded $\mathrm{C}R_x$ in terms of $\dyad{0}$ and $\dyad{1}$. Then, in order to proof by contradiction, we can assume that $Y = I_n \otimes Y'$, then $\tr_n(Y) = Y'$, so
  \begin{equation}
    Y' = \tr_n(Y) = \tr_n(X') \dyad{0}_n + \tr_n(X') \dyad{1}_n = \tr_n(X') I_t
  \end{equation}
  which means that $Y = \tr_n(X') \otimes I_n \otimes I_t$, that results in $X = \tr_n(X') \otimes I_n \otimes I_t$,  as well. Which is a contradiction with the assumption that $X$ is nontrivial on $n$.
\end{proof}

Now, we can prove the optimality of the bridged $T$ gate for class I gates in the theorem below.

\begin{theorem}\label{thm:no-go-class-i}
  Any bridged $T$ gate where $T$ is a class I gate acting on qubits $1$ and $n$ with linear connectivity, needs at least $4n$ CNOTs and has at least $n$ depth.
\end{theorem}
\begin{proof}
  Without losing generality, we can assume that $T = \mathrm{C}R_x^{1\to n}(\theta)$, because of Lemma~\ref{lem:crx-decomposition}. We can also show that the successor of $X_1$ and $Z_n$ under $T$ is nontrivial on $1$ and $n$, and this must also be valid for any sequence of gates that are equivalent to $T$.

  Then, we assume that a sequence of gates $A = A_\ell \dots A_2 A_1$ is equivalent to $T$, then, we name the successor of $X_1$ and $Z_n$ under $A_i \dots A_1$ as $\xi_i$ and $\zeta_i$ respectively. Now, based on the fact that one-qubit gates could not create nontrivial action (Corollary~\ref{cor:no-go-one-qubit}), and because of the connectivity constraint, in order to have nontrivial action on $n$ by $\xi_n$, we need to a CNOT gate between $n$ and $n-1$ and also $x_i$ for some $i \ne n$, should act nontrivially on $n-1$. Moreover, using Lemma~\ref{lem:no-move-class-i} we know that using one CNOT will leave $\xi_n$ acting nontrivially on $n-1$, which is not desired. This means that at least two CNOTs are necessary. By recursively applying this argument, we can conclude that a sequence of CNOT pairs acting on subsequent qubits from $1$ to $n$ is necessary to have nontrivial action on $n$ by $\xi_n$. A similar argument could hold for $\zeta_n$. Now because of the fact that these two sequences are going in opposite directions, and they cannot have more than $O(1)$ gates in common, we can conclude that at least $4n$ CNOTs are necessary to have nontrivial action on $1$ and $n$ by $\xi_n$ and $\zeta_n$. Note that the minimum depth could be easily derived from the necessity of a sequence of CNOTs from $1$ to $n$ earlier.
\end{proof}

\subsection{Class II}

For the class II, we will prove in a similar manner that there is no solution better than the naive solution with SWAPs. Here we start by a lemma that shows the distinctive feature of the class II gates to be used in the proof.

\begin{lemma}\label{lem:no-trivial-commutation}
  for any gate $X$ acting nontrivial on $n$ and trivial on $t$, its successor $Y$ under a class II gate $U$ acting on $n$ and $t$, is non-trivial on $t$. 
\end{lemma}
\begin{proof}
  Similar to the approach in proving Lemma~\ref{lem:no-move-class-i}, using proof by contradiction, we can assume that $Y = Y' \otimes I_t$, then
  \begin{equation}
    \begin{aligned}
      (X' \otimes I_t) U  &= U (Y' \otimes I_t) \\
      (X' V_1 \otimes V_2) e^{i \alpha X \otimes X + i \beta Y \otimes Y + i \gamma Z \otimes Z} (V_3 \otimes V_4) &= (V_1 \otimes V_2) e^{i \alpha X \otimes X + i \beta Y \otimes Y + i \gamma Z \otimes Z} (V_3 Y' \otimes V_4). 
    \end{aligned}
  \end{equation}
  Then by multiplying $(V_1^\dagger \otimes V_2^\dagger)$ from left and $(V_3^\dagger \otimes V_4^\dagger)$ from right, we can write
  \begin{equation}
    \begin{aligned}
      (V_1^\dagger X' V_1 \otimes I_t) e^{i \alpha X \otimes X + i \beta Y \otimes Y + i \gamma Z \otimes Z} &= e^{i \alpha X \otimes X + i \beta Y \otimes Y + i \gamma Z \otimes Z} (V_3 Y' V_3^\dagger \otimes I_t).  \\
      (V_1^\dagger X' V_1 \otimes I_t) (c_1 I \otimes I + i c_2 X \otimes X + i c_3 Y \otimes Y + i c_4 Z \otimes Z) &= (c_1 I \otimes I + i c_2 X \otimes X + i c_3 Y \otimes Y + i c_4 Z \otimes Z) (V_3 Y' V_3^\dagger \otimes I_t).  
    \end{aligned}
  \end{equation}
  Where $c_i$s are some constants. Finally, by multiplying both sides by $(I_n \otimes P)$ where $P$ is one of $I, X, Y, Z$, and then by tracing out the qubit $t$, we will have the following equations:
  \begin{equation}
    \begin{cases}
      V_1^\dagger X' V_1 = V_3 Y' V_3^\dagger \\
      V_1^\dagger X' V_1 X = X V_3 Y' V_3^\dagger & \text{if }\alpha \ne 0 \\
      V_1^\dagger X' V_1 Y = Y V_3 Y' V_3^\dagger & \text{if }\beta \ne 0 \\
      V_1^\dagger X' V_1 Z = Z V_3 Y' V_3^\dagger & \text{if }\gamma \ne 0. \\
    \end{cases}
  \end{equation}

  It means that if two of $\alpha, \beta, \gamma$ are non-zero, then $X'$ must have the form $X' = X'' \otimes I_n$ which is against the assumption that $X$ is nontrivial on $n$.
\end{proof}

Now, we can prove the main theorem.

\begin{theorem}\label{thm:no-go-class-ii}
  Any bridged gate where $T$ is a class II gate acting on qubits $1$ and $n$ with linear connectivity, needs at least $6n$ CNOTs and has at least $3n$ depth.
\end{theorem}
\begin{proof}
  This proof is built upon the proof of Theorem~\ref{thm:no-go-class-i}, we have already made an argument that if 
  { \color{red} TODO just to say that two is not enough because there is a unitary that the first CNOT misses, if the second CNOT is doing it, then it need a third CNOT to clean. }
\end{proof}
\subsection{Extension to Arbitrary Connectivity}

The extension to arbitrary connectivity is straightforward. Given a connectivity graph $G = (V, E)$, and a target gate $T$ that is defined on $a,b \in V$, we can define a set of layers $L_1 \dots L_k$, defined as
\begin{equation}
  L_i = \{ v \in V \mid \mathrm{dist}(v, a) = i \}.
\end{equation}
Where $\mathrm{dist}(v, a)$ is the distance between $v$ and $a$ in $G$. Then, the whole proof could be re used, with the only difference that instead of acting nontrivially on $i$th qubit, now we care about acting nontrivially on one of the qubits in $L_i$. This means that the number of CNOTs and the depth will be $4\mathrm{dist}(a,b) + O(1)$ and $\mathrm{dist}(a,b) + O(1)$ respectively. 

\section{The Ancilla Case}

As like as many other efforts, we assumed that any qubit in the device is used and is continuously carrying information. This might not be a bad idea in NISQ devices, where any single qubit is precious and the number of qubits is limited. However, because of its interesting results and its potential to be used in future devices, we will briefly discuss the case where we have ancilla qubits as well.

Bridging class I gates over $n$ ancilla qubit needs $2n + O(1)$ CNOTs and this number for class II gates is $4n + O(1)$, which shows a significant improvement in comparison to th previous results.
{ \color{red} TODO: show circuits for these cases 
  mention network coding as well \cite{ho2008}
}