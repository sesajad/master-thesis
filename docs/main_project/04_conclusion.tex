\chapter{Conclusion}\label{chap:conclusion}

In this essay, we have studied the problem of bridging that has not been directly addressed yet. We presented an optimal implementation for bridging class I gates, such as CNOT, that uses only $4n+O(1)$ CNOTs and depth $n+O(1)$. This outperforms previous bridging methods by $33\%$ in CNOT count and $66\%$ in depth. We also proved that the baseline SWAP implementation is optimal for bridging class II gates like SWAP, requiring $6n+O(1)$ CNOTs. Furthermore, we generalised these results to arbitrary connectivity graphs, showing the bridging complexity depends on the distance between qubits.

\section{Future works}

Here we discuss some possible future works that can be done in this field.

\subsection{Implementation}

None of existing quantum compilers have implemented bridging for more than one qubit in between yet. The implementation of bridging can be done as an alternative way of routing, with a decision criteria for choosing between the two methods, or as a simplification that is done after routing. The question of how-to-do and when-to-do here needs meaningful engineering efforts and will be interesting to see.

\subsection{Simultaneous Bridging}

One of the possible future works is to consider the case where we have multiple gates to be bridged simultaneously. A special case of this problem is when we have multiple SWAP gates on non-adjacent qubits that we want to do in the shortest time. This problem is approached in \cite{childs} by reducing it to \textsc{RoutingViaMatching} problem \cite{banerjee2017}. 

\subsection{Ancillas}

Bridging gates over ancilla qubits will be doubtlessly easier, as we should not worry about preserving an arbitrary state in the qubits in between. A simple case for CNOT and SWAP gates is shown in Figure \ref{fig:bridged-cnot-w-ancilla} and Figure \ref{fig:bridged-swap-ancilla} respectively. This situation might be related to the field of network coding \cite{ho2008}, that in simplest terms, aims to transmit information in a classical network made of bits, that are all initially zero.

\begin{figure}[h!]
  \label{fig:bridged-cnot-w-ancilla}
  \centering
  \begin{tikzpicture}
    \node[scale=0.7] {
    \begin{quantikz}
    \qw & \ctrl{2} & \qw \midstick[3,brackets=none]{=} &  & \ctrl{1} & \qw & \ctrl{1}& \qw \\
    \lstick{$\ket{0}$} \qw & \qw & \qw & \lstick{$\ket{0}$} & \targ{} & \ctrl{1}  & \targ{} & \qw \\
    \qw & \targ{} & \qw &  & \qw  & \targ & \qw  & \qw & \qw \\
  \end{quantikz}
  };
    \end{tikzpicture}
  \caption{A bridged CNOT gate over one ancilla qubit}
\end{figure}

\begin{figure}[h!]
  \label{fig:bridged-swap-ancilla}
  \centering
  \begin{tikzpicture}
  \node[scale=0.7] {
    \begin{quantikz}
      \qw & \swap{2} & \qw \midstick[3,brackets=none]{=} &
       & \ctrl{1} & \targ{} & \qw & \qw & \qw & \targ{} & \ctrl{1} & \qw \\
      \lstick{$\ket{0}$} \qw & \qw & \qw &
      \lstick{$\ket{0}$} & \targ{}& \ctrl{-1} & \ctrl{1} & \targ{} & \ctrl{1} & \ctrl{-1}& \targ{} & \qw \\
      \qw & \swap{} & \qw  &
       & \qw & \qw & \targ{} & \ctrl{-1} & \targ{} & \qw & \qw & \qw  & \qw \\
    \end{quantikz}
    };
    \end{tikzpicture}
  \caption{A bridged SWAP gate over one ancilla qubit}
\end{figure}

\section{Final Remark}

In conclusion, we have demonstrated asymptotically optimal constructions for bridging key two-qubit gate classes, with significantly reduced CNOT count and depth over prior methods. Our work helps enable more efficient routing in NISQ devices by avoiding explicit SWAPs. Further developing bridging techniques could provide a valuable tool for mitigating connectivity constraints in quantum architectures.