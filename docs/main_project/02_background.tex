\chapter{Background}\label{chap:background}

We should be concerned with two key findings from the literature review on quantum compilation. First, the way that they broke down the problem and the assumption they made to simplify the problem space and introducing some structure to that. On the other hand, is the alogrithms that they involve techniques from QC, graph theory and more.

Here we try to review both of these aspects to draw a big picture of our notion of quantum compilation and also to review the existing techniques related to our approach, like the special techniques for two-qubit gates.

\section{Quantum Compilation}
The term ``quantum compilation'' can refer to any process that transforms a higher-level description of a quantum algorithm into a lower-level description~\cite{hundt2022}. In the majority of works \cite{zulehner2018,childs,cross2022,sivarajah2021,qiskit2023,paler2021}, circuits are the description used and thereby the compilation is done by transforming a general quantum circuit into a circuit that is compatible with a specific hardware. Given that the problem of finding the most optimal circuit (with respect to a sense of complexity like depth or number of gates) is proven to be NP-hard \cite{siraichi2018}, the research primarily centers on deconstructing this problem into smaller components and devising techniques to effectively balance between the agility of the process and the quality of the solution. This mirrors the approach adopted in classical compilation \cite{allen2001}.

However, the clear structure for this breakdown has not been firmly established, with numerous diverse approaches in play. As such, our goal is to provide a comprehensive overview of the issue and identify common patterns in the existing literature. This overall picture will then serve as a reference point throughout the rest of this thesis. To achieve this overview, it's crucial to define \textbf{circuit transformation} as a process that preserves the essential meaning of the circuit, ensuring that the circuit's output remains consistent for any input before and after the transformation. Yet, this process alters the circuit to adhere to specific constraints or optimize it for particular goals.

Thus, \textbf{quantum compilation} in our essay will be a sequence of circuit transformations where each transformation either enforces a constraint or optimizes the circuit (where optimization itself can be perceived as a soft constraint). The primary constraints imposed upon the circuit arise from the characteristics of the hardware and are listed below:

\begin{itemize}
  \item \textbf{Gate set}: The set of gates physically available in the hardware.
  \item \textbf{Connectivity}: The connectivity between qubits in the hardware topology.
\end{itemize}
  
Furthermore, the optimizations that corresponds to different degrees various degrees of freedom (e.g., qubit assignments, choosing among equivalent subcircuits) can be applied. They are commonly pursuing these goals:

\begin{itemize}
  \item \textbf{Complexity reduction}: Reducing the number of gates or depth of the circuit.
  \item \textbf{Preparation for constraints}: Minimizing violation of the mentioned constraints before imposing them. 
  \item \textbf{Error mitigation}: The error of the circuit with respect to the hardware, especially in NISQ devices.
\end{itemize}

Quantum compilation frameworks mainly use a similar approach, they introduce transformations that are each is somehow seeking one of the goals defined before, tket even uses the term predicates for the constraints that are similar to these \cite{sivarajah2021}. Typical transformations in their compilation process are, decomposing gates (imposing gate set constraint), assigning qubits (optimization of connectivity as a soft constraint), routing (imposing connectivity constraint), and further optimizations (reducing complexity). Yet the order of these transformations is not the same everywhere, for example, some \cite{lao2021,qiskit2023} deffer the gate synthesis after the routing while some \cite{wille2020,sivarajah2021} does it otherwise.

With this background established, we now dive into details on gate set and connectivity constraints, and circuit optimizations techniques for reducing complexity.

\subsection{Gate synthesis}

Imposing the gate set constraints which is known as the gate synthesis, is one of the oldest subroutines in the quantum compilation. The problem here is to decompose a general $n$-qubit gate into a sequence of gates from the physically available gates on the device. Here we review the results for the synthesis of one, two, and more than two-qubit gates. 

While most of the devices have the capability to perform arbitrary one-qubit gates, with a neglectible cost, there is a famous result for the synthesis of one-qubit gates, called Solovay-Kitaev theorem \cite{dawson2006}. It states that any one-qubit gate can be approximated by a sequence of gates from a universal gate set with an error of $\epsilon$ using $O(\log^c(1/\epsilon))$ gates, where $c$ is a constant. Here as like as other results, we assume that all of one-qubit gates are available.

For two-qubit gates, it is known that the set of all one-qubit gates together with CNOT can produce any two-qubit gate and it can be done by upto three CNOTs (and it is optimal) \cite{vatan2004,vidal2004}. One of the famous gates that needs three CNOTs is SWAP. Using CNOT as the only available two-qubit gate although is common in theoretical studies of compilation \cite{zulehner2018,siraichi2018,li2019,zhang2021,zhou2020,itoko2019,murali2019} there are other continuous gate sets such as $\mathrm{fSim}(\theta, \phi)$ \cite{foxen2020} or other Hamiltonians evolution \cite{childsa}. Also, the evolution of XYZ Heisenberg interaction Hamiltonian, has been used as an intermediate gate set or a tool for analytical analysis in some works \cite{sousa2006,vidal2004}. The importance of this Hamiltonian is because of a decomposition (Cartan's KAK decomposition, Cirac-Kraus decomposition, or Khaneja-Glaser decomposition) that states any two-qubit gate can be made by only one evolution of $XYZ$ Hamiltonian and four one-qubit gates \cite{kraus2001,khaneja2001}.

For more than two-qubit gates, first TOFOLLI gate was used to prove the universality \cite{barenco1995}, later it was proven that it is not necessary. Although there are a few ways to synthesis a general $n$-qubit gate  \cite{sousa2006,shende2006}, they are not commonly used because of its inherent inefficiency. It is known that it could not be better than $O(4^n)$ gates. \cite{shende2006}

Synthesis of local Hamiltonian evolution, as a special case of $n$-qubit gate, is also studied as it is important for simluation purposes \cite{childs2018}. Most of the researchs are built upon Suzuki-Trotter decomposition~\cite{trotter1959, suzuki1991}, that approximates the time evolution of a Hamiltonian with a sequence of time evolutions of its terms. While most of the works rely on the first-order Suzuki-Trotter decomposition (ak Lie-Trotter formula) ~\cite{sivarajah2021, qiskit2023}, there are a few efforts to analysis higher-order errors~\cite{childs2021} or to use other decomposition, such as a random sampling decomposition (QDRIFT) ~\cite{campbell2019}. Beside the gate synthesis, Hamiltonian compilation has implications that can be used in other parts of the compilation process, for example ~\cite{lao2021} defines routing and scheduling algorithms specifically for Hamiltonian compilation.


\subsection{Qubit Allocation and Routing}

Connectivity constraint as the hardest constraint to impose, is the main focus of many researches. Research often break it down into two subproblems, qubit allocation and routing. The definition of them may vary in the literature, but we can roughly define the \textbf{qubit allocation} qubit allocation we roughly mean a mapping from logical qubits of a quantum circuit to physical qubits of a device in way that minimizes the circuit's complexity overhead due to routing. And by \textbf{routing} we mean the problem of finding an optimal circuit that is compatible with device connectivity and is equivalent to the circuit.

Qubit allocation shares common traits with other resource allocation problems \cite{alicherry2012},\cite[pp. 440-444]{allen2001} and it is not wondering that it is NP-hard for arbitrary connectivity graphs, this can be shown easily by a reduction from graph isomorphism~\cite{siraichi2018}. Note that real-world devices are not arbitrary graphs and it might be exactly solvable in reasonable time for certain families of graphs. 

There have been attempts to find the optimal solution by searching for arbitrary connectivity graphs, which can be feasible for small devices~\cite{siraichi2018}. But, the most common approach in the research is to use a heuristic~\cite{zhang2021, itoko2019, cowtan2019,paler2019, murali2019} together with a search algorithm (such as BFS, $A^*$\cite{zulehner2018}, simulated annealing\cite{zhou2020}, or others\cite{li2019}) to find a reasonable solution. Some of these efforts have also been implmented in current quantum compilers \cite{qiskit2023,sivarajah2021}.

% TODO, do we need more details? or more papers?

Results suggest that the qubit allocation could affect the complexity of the circuit by $10\%$ in realistic scenarios~\cite{paler2019}. This fact justifies such a break-down of connectivity constraint into two subproblems furthermore.

\subsection{Optimizations}

% code motion \cite[p. 592]{aho1986} similar to commutation relations \cite{TODO}

Circuit optimization is often achieved by applying simplification rules to the circuit~\cite{pointing2021}. These simplification rules are usually based on the commutation relations of the gates~\cite{itoko2019}.

Yet, this simplification rules are implemented as a pattern matching, therefore the hidden patterns that can be revealed by another simplification rules will not be found.

\section{Two-qubit gates}

% \sout{Papers with a focus on decomposition of gates and KAK} (maybe) \cite{tucci2005,vatan2004a}

% Papers with a focus on entangling power (maybe) \cite{nielsen2003,berry2002,bennett2002,linowski2020}

Another important fact about the gate synthesis of two-qubit gate, which is fruitful for many theoretical purposes is KAK (aka Khaneja-Glaser~\cite{khaneja2001} or Kraus-Cirac~\cite{kraus2001}) decomposition.??



\subsubsection{Kraus-Cirac Decomposition}

The Kraus-Cirac decomposition~\cite{kraus2001} helps us create arbitrary two-qubit gates. It states that any two-qubit gate can be created by a Heisenberg model and local gates.

\begin{theorem}[Kraus-Cirac Decomposition]
For any $U \in SU(4)$, there exist $V_1, V_2, V_3, V_4 \in SU(2)$ together with $\alpha, \beta, \gamma \in \mathbb{R}$ such that
\begin{equation}
  U = V_1 \otimes V_2 e^{\alpha X\otimes X + \beta Y\otimes Y + \gamma Z\otimes Z} V_3 \otimes V_4.
  \end{equation}
\end{theorem}

This theorem, together with the basic properties of entanglement power, will also lead to many more results on communication using two-qubit gates~\cite{berry2002} and also the universality and optimality of three CNOTs for two-qubit gates~\cite{vatan2004}.


\subsection{Entanglement Power}

We already know that there are so many measures for the entanglement of a bipartite state, building upon those measures, we can define entanglement power.

Entanglement power is a measure of the ability of a two-qubit gate to create entanglement. It has multiple definitions, such as the maximum amount of ebits that can be created from product states~\cite{shen2018}, or the average amount of them (with respect to a Haar distribution)~\cite{zanardi2000}, or even the number of terms in a Schmidt decomposition (which will be equal to the number of non-zero terms in the Kraus-Cirac decomposition)~\cite{nielsen2003}.

This measure, assigns a number to each gate, and then, by composing gates, the entanglement power could not exceed the summation of the entanglement powers of the gates that are used to create the gate. This fact is used to prove many tight bounds for decomposition of two-qubit gates.

Moreover, these efforts implicitly define a hierarchy of two-qubit gates based on the number of non-zero terms in the interaction. For example, CNOT has one non-zero term while SWAP has three.

To understand the current state of the art in quantum compilation, we will first look at the general compilation problem. 


\chapter{Related Works}\label{chap:discussion}

\subsubsection{Routing and Bridging}

% about relation to inital mapping (allocation)

% The compiler includes two different heuristic search algorithms, one for initial qubit allocation and another for subsequent qubit allocation. The algorithm for subsequent qubit allocation also provides the route (of SWAP gates) for reallocation. This compiler supports all one and two-qubit gates as intermediate gates, which helps simplify the circuit by unifying consecutive two-qubit gates~\cite{lao2021}.

In most cases, the treatment of the initial mapping and subsequent mappings is different~\cite{zhou2020, li2019}. This is because while subsequent mappings can be seen as a routing problem in the permutation space, the initial mapping is a search to find the best starting point.

Another approach is to use partial permutations, which allows for the use of the same algorithm for both initial mapping and subsequent mappings~\cite{zulehner2018}. % that old asf paper

\begin{definition}[Partial Mapping]
  A partial mapping, is a partial injective function from the logical qubits to the physical qubits.
  It means that some of the logical qubits may not be mapped to any physical qubit.
\end{definition}

% Papers with a focus on routing problem \cite{childs,zhou2020,itoko2019,cowtan2019}

Routing, by child:
However, real-world devices are not arbitrary, and by imposing some restrictions on the connectivity graph, we can solve the problem in polynomial time~\cite{childs}. For example, the problem is solvable in polynomial time for path graphs, complete graphs, tree graphs, and product graphs. These results has already been known for similar classical problems like token swapping and routing via matching~\cite{banerjee2017}.

\begin{problem}[Routing via Matching]
  Given a graph and a set of pebble, each of them at each node, a permutation is given and we need to achive the permutation by moving the pebbles. each move consists of swapping two pebbles at adjacent nodes. The problem is to find the minimum number of moves (a.k.a. routing number).
\end{problem}

% Ideas to improve SWAP complexity (embedding from 2QAN and CNOT framework from Kissinger) \cite{lao2021}\cite{nash2020,kissinger2019}

% Mention network coding as well \cite{ho2008}

The solution to the qubit allocation problem will not necessarily specify the SWAPs that are needed to change the mapping, although some approaches do so~\cite{childs, li2019, zhou2020}. In other cases, we need to use a routing or search algorithm to find the SWAPs that are required to change the mapping~\cite{zulehner2018, sivarajah2021}.

Moreover, bridge gates can be used as an alternative in cases where we need to SWAP back and forth between two qubits. While most papers use only bridge gates for three qubits~\cite{sivarajah2021,itoko2019,shende2006,siraichi2018} (one qubit in between), the general case of bridge gates is also studied~\cite{zhou2020, nash2020}.

\def\qceq{\midstick[3,brackets=none]{=}}

\begin{figure}[h]
  \label{fig:bridge-one-with-swap}
  \centering
\begin{tikzpicture}
\node[scale=0.7] {
  \begin{quantikz}
  \lstick{a} & \ctrl{2} & \qw \qceq & \swap{1} & \qw & \swap{1} & \qw\qceq & \ctrl{1} & \targ{} & \ctrl{1} & \qw &\ctrl{1} & \targ{} & \ctrl{1} & \qw \\
  \lstick{b} & \qw & \qw & \swap{} & \ctrl{1} & \swap{} & \qw & \targ{} & \ctrl{-1}& \targ{} & \ctrl{1} & \targ{} & \ctrl{-1}& \targ{} & \qw \\
  \lstick{c} & \targ{} & \qw  & \qw & \targ{} & \qw & \qw & \qw & \qw & \qw & \targ & \qw & \qw & \qw & \qw  & \qw \\
  \end{quantikz}
};
\end{tikzpicture}
  \caption{Applying a CNOT gate on $(a, c)$ using a SWAP gate}
\end{figure}

\begin{figure}[h]
  \label{fig:bridge-one-with-bridge}
  \centering
  \begin{quantikz}
  \lstick{a} & \ctrl{2} & \qw \qceq & \qw & \ctrl{1} & \qw & \ctrl{1} & \qw \\
  \lstick{b} & \qw & \qw & \ctrl{1} & \targ{} & \ctrl{1}  & \targ{} & \qw \\
  \lstick{c} & \targ{} & \qw & \targ{} & \qw  & \targ & \qw  & \qw &  \qw \\
  \end{quantikz}
  \caption{Applying a CNOT gate on $(a, c)$ using a bridge gate}
\end{figure}

Figure~\ref{fig:bridge-one-with-swap} and \ref{fig:bridge-one-with-bridge} show the difference between using a SWAP gate and a bridge gate to perform a CNOT gate for the case of three qubits. The bridge gate is more efficient in terms of the number of CNOT gates, but it is less efficient in terms of depth.

Now, the generalized bridge gate is defined as follows~\cite{nash2020}:

\begin{definition}{Generalized Bridge Gate}
  A CNOT between qubit $1$ and $n$ can be performed using a generalized bridge gate as follows:
  \begin{equation} \mathrm{Bridge}(1, n) = \prod_{i=1}^{n - 1}(\mathrm{CNOT}(i + 1, i) \prod_{i=n - 2}^{2}\mathrm{CNOT}(i + 1, i))^2
  \end{equation}
\end{definition}


\def\qceq{\midstick[6,brackets=none]{=}}
\begin{figure}[h]
  \centering
\begin{tikzpicture}
\node[scale=0.7] {
\begin{quantikz}
\qw &\ctrl{5}&\qw\qceq&
 \qw    &\qw     &\qw     &\qw     &\ctrl{1}& \qw    & \qw    & \qw     &
 \qw    &\qw     &\qw     &\qw     &\ctrl{1}& \qw    & \qw    & \qw     & \\
\qw & \qw    & \qw    &
 \qw    &\qw     &\qw     &\ctrl{1}& \targ{}&\ctrl{1}& \qw    & \qw     &
 \qw    &\qw     &\qw     &\ctrl{1}& \targ{}&\ctrl{1}& \qw    & \qw     & \\
\qw & \qw    & \qw    &
 \qw    &\qw     &\ctrl{1}& \targ{}& \qw    &\targ{} &\ctrl{1}& \qw     &
 \qw    &\qw     &\ctrl{1}& \targ{}& \qw    &\targ{} &\ctrl{1}& \qw     & \\
\qw & \qw    & \qw    &
 \qw    &\ctrl{1}&\targ{} & \qw    & \qw    & \qw    &\targ{} &\ctrl{1} &
 \qw    &\ctrl{1}&\targ{} & \qw    & \qw    & \qw    &\targ{} &\ctrl{1} & \\
\qw & \qw    & \qw    &
\ctrl{1}&\targ{} &\qw     & \qw    & \qw    & \qw    & \qw    &\targ{}  &
\ctrl{1}&\targ{} &\qw     & \qw    & \qw    & \qw    & \qw    &\targ{}  & \\
\qw &\targ{} & \qw    &
\targ{} & \qw    & \qw    & \qw    & \qw    & \qw    & \qw     & \qw    &
\targ{} & \qw    & \qw    & \qw    & \qw    & \qw    & \qw     & \qw    & \\
\end{quantikz}
};
\end{tikzpicture}
  \caption{The bridge gate for $n=6$}
\end{figure}
