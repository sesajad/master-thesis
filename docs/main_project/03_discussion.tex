\chapter{Discussion}\label{chap:discussion}

\section{Problem Statement}

We have already reviewed the main research themes around quantum compilation in Chapter~\ref{chap:background}. We have seen that one of the main challenges is in imposing connectivity constraints on the circuit. Furthermore, because of the huge dependence of most of the research on swapping qubits, putting together with the fact that swapping is the most expensive two-qubit gate, it is important to study the alternatives for swapping. In this chapter, we introduce our research problem based on the definition of bridging that we discussed earlier.

While our aim in the most broad term is to find alternatives to SWAP gate, to apply gates with respect to connectivity constraints, we start with a more specific problem. We seek for optimal bridged gate, and by optimal we mean both the least number of CNOTs and the least depth. Note that, this problem is going to be studied for an arbitrary connectivity constraint $G$.

The relation of this problem to the approach of \cite{nash2020,kissinger2019} is that we are looking for bridged gates that can be adopted in the any compilation process, in contrast to theirs that is defining a whole new compilation process. In terms of the results, while some of our preliminary results such as a simple bridged CNOT gate can be derived from their techniques too, however, the main contribution of our work is to show the optimal bridged gate for any two-qubit gate, while those efforts are limited to compiling the gates that are combination of CNOTs and $R_z$s.

\section{Two-qubit Gate Classes}

Our discussion heavily relies on the KAK decomposition of two-qubit gates. We have already introduced it in Lemma~\ref{lem:kak_decomposition}. Here we introduce a few classes of two-qubit gates, based on their KAK decomposition.

\begin{definition}[Two-qubit Gate Classes]
  Based on the KAK decomposition of two-qubit gates, that is 
  \begin{equation}
    U = (V_1 \otimes V_2) e^{i \alpha X \otimes X + i \beta Y \otimes Y + i \gamma Z \otimes Z} (V_3 \otimes V_4),
  \end{equation}
  we define the following classes of two-qubit gates:
  \begin{itemize}
    \item \textbf{Class nil}: Two-qubit gates that $\alpha = \beta = \gamma = 0$.
    \item \textbf{Class I:} Two-qubit gates that only one of $\alpha, \beta, \gamma \neq 0$.
    \item \textbf{Class II:} Two-qubit gates that at least two of $\alpha, \beta, \gamma \neq 0$.
  \end{itemize}
\end{definition}

The gates in class nil are local gates that are applied on each qubit separately, so they are irrelevant to our discussion about bridging. The gates in class I, such as CNOT, CZ and etc., can be decomposed to one $\mathrm{C}R_x$ gate upto local gates. This fact is shown in Lemma~\ref{lem:crx-decomposition} and enables us to bridge them.

\begin{lemma}\label{lem:crx-decomposition}
  For any two-qubit gate $U$ in class I, it can be decomposed as 
  \begin{equation}
    U = W_1 \otimes W_2 \mathrm{C}R_x(\theta) W_3 \otimes W_4,
  \end{equation}
  where $\mathrm{C}R_x(\theta)$ is a controlled rotation gate around $x$ axis, with control on the first qubit and target on the second qubit.
\end{lemma}
\begin{proof}
  Without lose of generality, we can assume that $\gamma \neq 0$ and $\alpha = \alpha = 0$. Now, we can write
  \begin{equation}
    U = (V_1 \otimes V_2) e^{i \gamma Z \otimes Z} (V_3 \otimes V_4).
  \end{equation}
  Then, similar to the way that we derive $\mathrm{C}Z$ and $\mathrm{CNOT}$ gates, we can write
  \begin{equation}
    \begin{aligned}
    U &= (V_1 \otimes V_2 R_z(-2\gamma)) CR_z(4\theta) (V_3 \otimes V_4), \\
    &= (V_1 \otimes V_2 R_z(-2\gamma) H) CR_x(4\theta) (V_3 \otimes H V_4).
    \end{aligned}
  \end{equation}
  Now we $W_i$s are easily derived from $V_i$s.
\end{proof}

The gates in class II, such as SWAP, cannot be bridged any better than the baseline implementation mentioned in Corollary~\ref{cor:baseline-bridged}. This fact is shown later using an special trait of these gates in Lemma~\ref{lem:no-trivial-commutation}. 

\section{Bridged Gates}

Here we introduce the bridged $T$ gate for any $T$ that belongs to class I. This gate is shown under a linear connectivity constraints, so it can be used for any other connectivity constraint (by using the connecting path between the qubits). An example of this gate (bridging over $4$ qubits) is shown in Figure~\ref{fig:bridged-class-i}. Note that $V_i$s and $\theta$ can be computed using Lemma~\ref{lem:crx-decomposition}

\begin{figure}[h!]
  \label{fig:bridged-class-i}
  \centering
  \begin{tikzpicture}
    \node[scale=0.7] {
\begin{quantikz}[transparent]
  \qw &\gate[6]{T}                        &\qw\midstick[6,brackets=none]{=}&
  \qw &\gate{V_3} & \ctrl{5} & \gate{V_1} &\qw\midstick[6,brackets=none]{=}&
  \gate{V_3} & \targ{}  & \qw     &\ctrl{1}& \qw    & \qw    & \qw    &\ctrl{1}& \qw     &\targ{}&\gate{V_1}\\
  \qw &\linethrough &\qw &
  \qw &\qw &\qw &\qw &\qw &
  \qw & \ctrl{-1}&\targ{}  & \targ{}&\ctrl{1}& \qw    &\ctrl{1}&\targ{} &\targ{}  &\ctrl{-1}&\qw\\
  \qw &\linethrough &\qw &
  \qw &\qw &\qw &\qw &\qw &
  \qw & \qw &\ctrl{-1}& \qw    & \targ{}&\ctrl{1}&\targ{} & \qw    &\ctrl{-1}&\qw & \qw \\
  \qw &\linethrough &\qw &
  \qw &\qw &\qw &\qw &\qw &
  \qw & \qw &\targ{}  & \qw    &\ctrl{1}& \gate{R_x(\theta)}&\ctrl{1}& \qw    &\targ{}  &\qw & \qw\\
  \qw &\linethrough &\qw &
  \qw &\qw &\qw &\qw &\qw &
  \qw & \targ{}  &\ctrl{-1}&\ctrl{1}& \targ{}& \qw    &\targ{} &\ctrl{1}&\ctrl{-1}&\targ{}&\qw \\
  \qw &  & \qw &
  \qw &\gate{V_4} & \gate{R_x(\theta)} & \gate{V_2} & \qw&
  \gate{V_4} & \ctrl{-1}& \qw     & \targ{}& \qw    & \qw    & \qw    &\targ{} & \qw     &\ctrl{-1}& \gate{V_2} 
  \end{quantikz}
  };
  \end{tikzpicture}
  \caption{A bridged class I gate over 6 qubits}
\end{figure}
  
This circuit above can be mathematically defined as follows:

\begin{theorem}[Bridged class I gate]
  For any class I two-qubit gate $T$, A bridged $T$ gate defined on qubits $1, n \in V$, where every $(i, i + 1) \in E$, can be implemented by the following sequence of gates:
  \begin{equation}
    T = (V_1 \otimes V_2) B^\dagger_n \mathrm{C}R_x^{\lceil n/2 \rceil \to \lceil n/2\rceil+1}(\theta) B_n (V_3 \otimes V_4),
  \end{equation}
  where $V_i$s and $\theta$ are defined in Lemma~\ref{lem:crx-decomposition}, and $B_n$ for $n = 4$ is defined as follows:
  \begin{equation}
    B_4 = (\CNOT^{1 \to 2} \otimes \CNOT^{3 \to 4}) (\CNOT^{2 \to 1} \otimes \CNOT^{4 \to 3}),
  \end{equation}
  and for an even $n \ge 6$ it could be written as follows:
  \begin{equation}
    \begin{aligned}
    B_{n} = &(\CNOT^{n/2 - 1 \to n/2} \otimes \CNOT^{n/2 + 1 \to n/2 + 2}) \\
    &(\CNOT^{n/2 - 2 \to n/2 - 1} \otimes \CNOT^{n/2 + 2 \to n/2 + 3}) \\
    &\prod_{i=1}^{n/2 - 3} (\CNOT^{i \to i+1} \otimes \CNOT^{i+3 \to i+2} \otimes \CNOT^{n-i-1 \to n-i-2} \otimes \CNOT^{n-i \to n-i+1}) \\
    &(\CNOT^{3 \to 2} \otimes \CNOT^{n-1 \to n-2}) \\
    &(\CNOT^{2 \to 1} \otimes \CNOT^{n \to n-1}).
    \end{aligned}
  \end{equation}
  and, for any odd $n$ it could be recovered by removing $n$th qubit and its corresponding CNOTs from $B_n$.
\end{theorem}
\begin{proof}
  In order to prove the correctness of this circuit, we will show that it is equivalent to the baseline implementation of bridged $T$ gate (Corollary~\ref{cor:baseline-bridged}).
  
  First, using the fact that $[\CNOT^{i \to j}, \CNOT{i \to k}] = 0$ and $[\CNOT^{i \to j}, \CNOT^{k \to j}] = 0$, we can show that $B_n$ (for even $n$) could be rearranged as:

  \begin{equation}
    B_n = \prod_{i=1}^{n/2 - 1} (\CNOT^{i\to i+1}\CNOT{i+1\to i}) \otimes (\CNOT^{n-i\to n-i+1} \CNOT{n-i+1\to n-i}).
  \end{equation}

  This rearrangement for $n = 6$ is shown in Figure~\ref{fig:bridged-class-i-proof-a}.

  \begin{figure}[h!]
    \label{fig:bridged-class-i-proof-a}
    \centering
    \begin{tikzpicture}
      \node[scale=0.6] {
  \begin{quantikz}[transparent]
    \gate{V_3} & \targ{}  & \qw     &\ctrl{1}& \qw    & \qw    & \qw    &\ctrl{1}& \qw     &\targ{}&\gate{V_1}&\qw\midstick[6,brackets=none]{=}&
    \gate{V_3} & \targ{}  &\ctrl{1}& \qw     & \qw    & \qw    & \qw    & \qw     &\ctrl{1}&\targ{}&\gate{V_1}& \qw \\
    \qw & \ctrl{-1}&\targ{}  & \targ{}&\ctrl{1}& \qw    &\ctrl{1}&\targ{} &\targ{}  &\ctrl{-1}&\qw&\qw&
    \qw & \ctrl{-1}&\targ{}  & \targ{}&\ctrl{1}& \qw    &\ctrl{1}&\targ{} &\targ{}  &\ctrl{-1}&\qw&\qw\\
    \qw & \qw &\ctrl{-1}& \qw    & \targ{}&\ctrl{1}&\targ{} & \qw    &\ctrl{-1}&\qw & \qw & \qw &
    \qw & \qw & \qw    &\ctrl{-1}& \targ{}&\ctrl{1}&\targ{} &\ctrl{-1}& \qw    &\qw & \qw & \qw \\
    \qw & \qw &\targ{}  & \qw    &\ctrl{1}& \gate{R_x(\theta)}&\ctrl{1}& \qw    &\targ{}  &\qw & \qw&\qw&
    \qw & \qw   & \qw    &\targ{}&\ctrl{1}& \gate{R_x(\theta)}&\ctrl{1}&\targ{}& \qw      &\qw & \qw&\qw&\\
    \qw & \targ{}  &\ctrl{-1}&\ctrl{1}& \targ{}& \qw    &\targ{} &\ctrl{1}&\ctrl{-1}&\targ{}&\qw&\qw& 
    \qw & \targ{}  &\ctrl{1}&\ctrl{-1}& \targ{}& \qw    &\targ{} &\ctrl{-1}&\ctrl{1}&\targ{}&\qw&\qw\\
    \gate{V_4} & \ctrl{-1}& \qw     & \targ{}& \qw    & \qw    & \qw    &\targ{} & \qw     &\ctrl{-1}& \gate{V_2} &\qw&
    \gate{V_4} & \ctrl{-1}& \targ{} & \qw  & \qw    & \qw    & \qw     & \qw     &\targ{}&\ctrl{-1}& \gate{V_2} &\qw&
    \end{quantikz}
    };
    \end{tikzpicture}
    \caption{Rearrangement of CNOTs that is used in the proof}
  \end{figure}

  Then, using the fact that 
  \begin{equation}
    \SWAP^{i,j} = \CNOT^{j \to i} \CNOT^{i \to j} \CNOT^{j \to i},
  \end{equation}
  it could be simplified further to
  \begin{equation}
    B_n = \prod_{i=1}^{n/2 - 1} (\CNOT^{i+1 \to i} \SWAP^{i, i+1}) \otimes (\CNOT{n-i+1\to n-i}\SWAP^{n-i, n-i+1}).
  \end{equation}

  Now, by moving the rearranging the CNOTs to be applied after the SWAPs, we will have
  \begin{equation}
    B_n = \prod_{i=1}^{n/2 - 1} \CNOT^{n/2 \to i} \otimes \CNOT^{n-i+1 \to n/2 + 1} \prod_{i=1}^{n/2 - 1} \SWAP^{i, i+1} \otimes \SWAP^{n-i, n-i+1}.
  \end{equation}

  This process is shown in Figure~\ref{fig:bridged-class-i-proof-b} for $n = 6$.
  
  \begin{figure}[h!]
    \label{fig:bridged-class-i-proof-b}
    \centering
    \begin{tikzpicture}
      \node[scale=0.6] {
  \begin{quantikz}[transparent]
    \gate{V_3} & \swap{}  &\targ{}& \qw     & \qw    & \qw    & \qw    & \qw     &\targ{}&\swap{}&\gate{V_1}&\qw\midstick[6,brackets=none]{=}&
    \gate{V_3} & \swap{}  & \qw     &\targ{}& \qw    & \qw    & \qw    &\targ{}& \qw     &\swap{}&\gate{V_1}&\qw \\
    \qw & \swap{-1}&\ctrl{-1}  & \swap{}&\targ{}& \qw    &\targ{}&\swap{} &\ctrl{-1}  &\swap{-1}&\qw&\qw&
    \qw & \swap{-1}& \swap{}&\qw&\targ{}& \qw    &\targ{}&\qw &\swap{} &\swap{-1}&\qw&\qw\\
    \qw & \qw & \qw    &\swap{-1}& \ctrl{-1}&\ctrl{1}&\ctrl{-1} &\swap{-1}& \qw    &\qw & \qw & \qw &
    \qw & \qw &\swap{-1}& \ctrl{-2} & \ctrl{-1}&\ctrl{1}&\ctrl{-1} &\ctrl{-2}& \swap{-1}    &\qw & \qw & \qw\\
    \qw & \qw   & \qw    &\swap{1}&\targ{}& \gate{R_x(\theta)}&\targ{}&\swap{1}& \qw      &\qw & \qw&\qw&
    \qw & \qw   & \swap{1}    &\targ{}&\targ{}& \gate{R_x(\theta)}&\targ{}&\targ{}& \swap{1}      &\qw & \qw&\qw\\
    \qw & \swap{1}  &\targ{}&\swap{}& \ctrl{-1}& \qw    &\ctrl{-1} &\swap{}&\targ{}&\swap{1}&\qw&\qw& 
    \qw & \swap{1}  &\swap{}&\qw& \ctrl{-1}& \qw    &\ctrl{-1} &\qw&\swap{}&\swap{1}&\qw&\qw\\
    \gate{V_4} & \swap{}& \ctrl{-1} & \qw  & \qw    & \qw    & \qw     & \qw     &\ctrl{-1}&\swap{}& \gate{V_2} &\qw&
    \gate{V_4} & \swap{}&  \qw & \ctrl{-2} & \qw    & \qw    & \qw     &\ctrl{-2}& \qw     &\swap{}& \gate{V_2} &\qw
    \end{quantikz}
    };
    \end{tikzpicture}
    \caption{Moving CNOTs across the SWAPs that is used in the proof}
  \end{figure}

  Finally, the second product term is similar to the one in Corollary~\ref{cor:baseline-bridged}, and the first product term is a sequence of CNOTs that all commute with $\mathrm{C}R_x^{n/2 \to n/2 + 1}(\theta)$, so they will be canceled in $B_n^\dagger \mathrm{C}R_x^{n/2 \to n/2 + 1}(\theta) B_n$.

  \begin{equation}
    \begin{aligned}
    B^\dagger_n \mathrm{C}R_x^{n/2 \to n/2 + 1}(\theta) B_n &= \prod_{i=n/2 - 1}^{1} \SWAP^{i, i+1} \otimes \SWAP^{n-i, n-i+1} \\
    & \phantom{=}~\mathrm{C}R_x^{n/2 \to n/2 + 1}(\theta) \\
    & \phantom{=}~\prod_{i=1}^{n/2 - 1} \SWAP^{i, i+1} \otimes \SWAP^{n-i, n-i+1} \\
    &= \mathrm{C}R_x^{1 \to n}(\theta)
    \end{aligned}
  \end{equation}

  Finally, by recalling Lemma~\ref{lem:crx-decomposition},
  \begin{equation}
    (V_1 \otimes V_2) B^\dagger_n \mathrm{C}R_x^{\lceil n/2 \rceil \to \lceil n/2\rceil+1}(\theta) B_n (V_3 \otimes V_4) = T.
  \end{equation}
  
  In order to prove the correctness of the circuit for odd $n$ as well, we can simply remove the $n$th qubit and its corresponding CNOTs from $B_n$. This fact will be propagated through the proof and every other thing will be the same.
\end{proof}

The formula above shows that the circuit uses $4n + O(1)$ CNOTs with the depth $n + O(1)$ (noting that $\mathrm{C}R_x$ needs two CNOTs \cite[chapter 4]{nielsen2010}). 
While in comparison to the baseline implementation (see Corollary~\ref{cor:baseline-bridged}), this circuit has $33\%$ improvement in the number of CNOTs and $66\%$ improvement in the depth. Even for the special case of CNOT, this circuit has approximately the same number of CNOTs as the bridged CNOT gate in Equation~\ref{eq:bridged-cnot-m-n} and Figure~\ref{fig:bridged-cnot-m-n}, however, its depth is improved by $75\%$.

\section{Proof of Optimality}

Here we aim to proof the optimality of the bridged $T$ gate described in the previous section for class I gates, and the baseline implementation for class II gates. By optimality we mean that the number of CNOTs and the depth of the circuit is the minimum possible (upto an $O(1)$ constant).

We start by introducing the the tools and the intuition behind their definitions, in addition to the core lemma that is used in the proofs. Then, we will prove the results for class I and class II gates separately. The first tool that we need is conjugation that is commonly used in group theory \cite{weisstein}.

\begin{definition}[Conjugation]
  For any two gates $A$ and $B$, we define the conjugation of $A$ by $B$ as
  \begin{equation}
    B^\dagger A B.
  \end{equation}
\end{definition}

The intuition behind the conjugation of gates is that if the conjugation of $A$ by $B$ is $C$, then applying $A$ before $A$ is equivalent to applying $C$ after $B$. Informally this could be phrased as by crossing $A$ over $B$, $A$ will be transformed to $C$. Figure~\ref{fig:conjugations} shows a few conjugations that we will use in the proofs.

\begin{figure}[ht]\label{fig:conjugations}
  \centering
  \begin{tikzpicture}
    \node at (-3, 0.4) {a)};
    \node[scale=0.7] {\begin{quantikz}
    \qw & \gate{X} & \ctrl{1} & \qw\midstick[2,brackets=none]{=} & \qw & \ctrl{1} & \gate[2]{U} & \qw \\
    \qw & \qw & \gate{R_x} & \qw & \qw & \gate{R_x} & \qw & \qw 
  \end{quantikz}};\end{tikzpicture} \\
  \begin{tikzpicture}
    \node at (-3, 0.4) {b)};
    \node[scale=0.7] {\begin{quantikz}
    \qw & \gate{Z} & \ctrl{1} & \qw\midstick[2,brackets=none]{=} & \qw & \ctrl{1} & \gate{Z} & \qw \\
    \qw & \qw & \gate{R_x} &  & \qw & \gate{R_x} & \qw & \qw 
  \end{quantikz}};\end{tikzpicture} \\
  \begin{tikzpicture}
    \node at (-3, 0.4) {c)};
    \node[scale=0.7] {\begin{quantikz}
    \qw & \gate{U} & \swap{1} & \qw\midstick[2,brackets=none]{=} & \qw & \swap{1} & \qw &  \qw \\
    \qw & \qw & \swap{} & \qw & \qw & \swap{} & \gate{U} & \qw 
  \end{quantikz}};\end{tikzpicture}
  \caption{a) Conjugation of $X \otimes I$ by $\mathrm{C}R_x$ is a two-qubit gate. b) Conjugation of $Z \otimes I$ by $\mathrm{C}R_x$ is $Z \otimes I$. c) Conjugation of $U \otimes I$ by SWAP is $I \otimes U$}.
\end{figure}

Furthermore, this effect, will help us to develop a tool to properly analyse the flow of information in a circuit. To complete this idea we define a sense of locality, using the following simple definition:

\begin{definition}[Trivial/Nontrivial action]
  A gate $U$ defined on a set of qubits $Q$ is acting trivially on qubit $q \in Q$ if $U$ can be written as 
  \begin{equation}
    U = U' \otimes I_q
  \end{equation}
  where $U'$ is a gate defined on $Q \setminus \{ q \}$ and $I_q$ is the identity gate on qubit $q$.
  Respectively, $U$ is acting nontrivially on $q$ if it cannot be written in the above form.
\end{definition}

Now, these proofs are based on the idea that if two gate sequence $A = A_n \dots A_2 A_1$ and $B = B_m \dots B_2 B_1$ are equal, then the conjugation of any arbitrary gate $U$ by $A$ is equal to the conjugation of $U$ by $B$. Or in another word,

\begin{corollary}
  Assume $A = A_n \dots A_2 A_1$ and $B = B_m \dots B_2 B_1$ are two gate sequences, then the necessary condition for $A = B$ is that for any arbitrary gate $U$, the conjugation of $U$ by $A$ must be equal to the conjugation of $U$ by $B$.
\end{corollary}

This corollary will be amazingly useful by noting the fact that the conjugation of $U$ by $A$ can be iteratively computed as follows:

\begin{equation}
  \begin{cases}
    U_1 = \text{conjugation of } U \text{ by } A_1 \\
    U_{i+1} = \text{conjugation of } U_i \text{ by } A_{i+1}
  \end{cases}
  \Rightarrow U_n = \text{conjugation of } U \text{ by } A.
\end{equation}

Now, we can start proving the optimality of the bridged $T$ gate for different classes, starting by class I.

\subsection{Class I}

The sketch of this proof consists of first reducing the problem to bridging a $\mathrm{C}R_x^{1\to n}$ gate. Then because of the nontrivial (on $1$ and $n$) conjugations of $X_1$ and $Z_n$ by $\mathrm{C}R_x^{1\to n}$, we can conclude that any sequence of gates that is bridged $\mathrm{C}R_x^{1\to n}$ needs to have a minimum number of CNOTs, and a minimum depth, to produce such nontrivial conjugations. These two bounds are achieved by the bridged $T$ gate.

So, to start the proof, we first establish two facts about the conjugations:
\begin{corollary}[No-go for one-qubit gates]\label{cor:no-go-one-qubit}
  For any gate acting trivially on $t$, its conjugation by any one-qubit gate on $t$ is still trivial on $t$. 
\end{corollary}

This corollary informally states that if we want to create nontrivial action on $t$ by conjugation, we need use two-qubit gates. This fact can be easily shown by simply combining definition of trivial action and conjugation.

\begin{lemma}[No-move for class I gates]\label{lem:no-move-class-i}
  For any gate $A$ acting nontrivially on $n$ and trivially on $t$, its conjugation $B$ by any class I gate $U$ acting on $n$ and $t$, is nontrivial on $n$.
\end{lemma}
\begin{proof}
  Noting that $A$ must be trivial on $t$ and nontrivial on $n$, $A$ could be decomposed as $A' \otimes I_t$.
  Using the definition of conjugation in form of $U^\dagger A U = B$, we can write

  \begin{equation}
    \begin{aligned}
    B &= U^\dagger (A' \otimes I_t) U  \\
    &= (V_3^\dagger \otimes V_4^\dagger)\mathrm{C}R_x(-\theta) (V_1^\dagger A' V_1 \otimes I) \mathrm{C}R_x(\theta) (V_3 \otimes V_4) \\
    &= (V_3^\dagger \otimes V_4^\dagger)(\dyad{0} + R_x(-\theta) \otimes \dyad{1}) (V_1^\dagger A' V_1 \otimes I) (\dyad{0} + R_x(\theta) \otimes \dyad{1}) (V_3 \otimes V_4) \\
    &= (V_3^\dagger \otimes V_4^\dagger)(V_1^\dagger A' V_1 \dyad{0} + R_x(-\theta) V_1^\dagger A' V_1 R_x(\theta) \dyad{1}) (V_3 \otimes V_4)
    \end{aligned}
  \end{equation}

  We have used the decomposition of Lemma~\ref{lem:crx-decomposition} for $U$ and expanded $\mathrm{C}R_x$ in terms of $\dyad{0}$ and $\dyad{1}$. Then, in order to proof by contradiction, we can assume that $B = I_n \otimes B'$, then $\tr_n(B) = B'$, so
  \begin{equation}
    B' = \tr_n(B) = \tr_n(A') \dyad{0}_n + \tr_n(A') \dyad{1}_n = \tr_n(A') I_t
  \end{equation}
  which means that $B = \tr_n(A') \otimes I_n \otimes I_t$, that results in $A = \tr_n(A') \otimes I_n \otimes I_t$,  as well. Which is a contradiction with the assumption that $A$ is nontrivial on $n$.
\end{proof}

Now, we can prove the optimality of the bridged $T$ gate for class I gates in the theorem below.

\begin{theorem}\label{thm:no-go-class-i}
  Any bridged $T$ gate where $T$ is a class I gate acting on qubits $1$ and $n$ with linear connectivity, needs at least $4n + O(1)$ CNOTs and has at least $n$ depth.
\end{theorem}
\begin{proof}
  Without losing generality, we can assume that $T = \mathrm{C}R_x^{1\to n}(\theta)$, because of Lemma~\ref{lem:crx-decomposition}. We can also show that the conjugation of $X_1$ and $Z_n$ (one qubit gates acting on qubits $1$ and $n$ respectively) by $T$ is nontrivial on $1$ and $n$ Then, we assume that a sequence of gates $A = A_\ell \dots A_2 A_1$ is equivalent to $T$, 
  the conjugation of $X_1$ and $Z_n$ under $A$ must be nontrivial on $1$ and $n$ as well. 

  In order to simplify the notation, we name the conjugation of $X_1$ and $Z_n$ by $A_i \dots A_1$ as $\xi_i$ and $\zeta_i$ respectively. Now, based on the fact that one-qubit gates could not create nontrivial action (Corollary~\ref{cor:no-go-one-qubit}), and because of the connectivity constraint, in order to have nontrivial action on $n$ by $\xi_n$, we need to a CNOT gate acting on $n$ and $n-1$, or in another word $A_j = \CNOT^{n \rightleftharpoons n-1}$ for some $j \le \ell$. Also $\xi_{j - 1}$, must act nontrivially on $n-1$. Moreover, using Lemma~\ref{lem:no-move-class-i} we know that using one CNOT ($A_j$) will leave $\xi_{j}$ acting nontrivially on $n-1$, which is not desired. This means that at least two CNOTs are necessary, or in another word $A_k = \CNOT^{n-1 \rightleftharpoons n-1 \pm 1}$ for $j < k \le \ell$. By recursively applying this argument, we can conclude that a sequence of CNOT pairs acting on subsequent qubits from $1$ to $n$ is necessary solely to make $\xi_n$ to have nontrivial action on $n$. A similar argument could hold for $\zeta_n$. Now because of the fact that these two sequences are going in opposite directions, and they cannot have more than $O(1)$ gates in common, we can conclude that at least $4(n-1) + O(1)$ CNOTs are necessary to have nontrivial action on $1$ and $n$ by $\xi_n$ and $\zeta_n$. Note that the minimum depth could be easily derived from the fact that a CNOT between $n$ and $n-1$ should take place after a CNOT between $n-1$ and $n-2$ and so on.
\end{proof}

\subsection{Class II}

For the class II, we will prove in a similar manner that there is no solution better than the baseline implementation (Corollary~\ref{cor:baseline-bridged}). Here we start by a lemma that shows the distinctive feature of the class II gates to be used in the proof.

\begin{lemma}\label{lem:no-trivial-commutation}
  for any gate $A$ acting nontrivially on $n$ and trivially on $t$, its conjugation $B$ by a class II gate $U$ acting on $n$ and $t$, is nontrivial on $t$. 
\end{lemma}
\begin{proof}
  Similar to the approach in proving Lemma~\ref{lem:no-move-class-i}, using proof by contradiction, we can assume that $B = B' \otimes I_t$, then
  \begin{equation}
    \begin{aligned}
      (A' \otimes I_t) U  &= U (B' \otimes I_t) \\
      (A' V_1 \otimes V_2) e^{i \alpha X \otimes X + i \beta Y \otimes Y + i \gamma Z \otimes Z} (V_3 \otimes V_4) &= (V_1 \otimes V_2) e^{i \alpha X \otimes X + i \beta Y \otimes Y + i \gamma Z \otimes Z} (V_3 B' \otimes V_4). 
    \end{aligned}
  \end{equation}
  Then by multiplying $(V_1^\dagger \otimes V_2^\dagger)$ from left and $(V_3^\dagger \otimes V_4^\dagger)$ from right, we can write
  \begin{equation}
      (V_1^\dagger A' V_1 \otimes I_t) e^{i \alpha X \otimes X + i \beta Y \otimes Y + i \gamma Z \otimes Z} = e^{i \alpha X \otimes X + i \beta Y \otimes Y + i \gamma Z \otimes Z} (V_3 B' V_3^\dagger \otimes I_t).
  \end{equation}

  Now, by expanding the exponential, we will have
  \begin{equation}
    \begin{aligned}
      (V_1^\dagger A' V_1 \otimes I_t) & (c_1 I \otimes I + i c_2 X \otimes X + i c_3 Y \otimes Y + i c_4 Z \otimes Z) \\
      = &(c_1 I \otimes I + i c_2 X \otimes X + i c_3 Y \otimes Y + i c_4 Z \otimes Z) (V_3 B' V_3^\dagger \otimes I_t).  
    \end{aligned}
  \end{equation}
  Where $c_i$s are some constants. Finally, by multiplying both sides by $(I_n \otimes P)$ where $P$ is one of $I, X, Y, Z$, and then by tracing out the qubit $t$, we will have the following equations:
  \begin{equation}
    \begin{cases}
      V_1^\dagger A' V_1 = V_3 B' V_3^\dagger \\
      V_1^\dagger A' V_1 X = X V_3 B' V_3^\dagger & \text{if }\alpha \ne 0 \\
      V_1^\dagger A' V_1 Y = Y V_3 B' V_3^\dagger & \text{if }\beta \ne 0 \\
      V_1^\dagger A' V_1 Z = Z V_3 B' V_3^\dagger & \text{if }\gamma \ne 0. \\
    \end{cases}
  \end{equation}

  It means that if two of $\alpha, \beta, \gamma$ are non-zero, then $A$ must have the form $A' = A'' \otimes I_n$ which is against the assumption that $A$ is nontrivial on $n$.
\end{proof}

Now, we can prove the main theorem.

\begin{theorem}\label{thm:no-go-class-ii}
  Any bridged gate where $T$ is a class II gate acting on qubits $1$ and $n$ with linear connectivity, needs at least $6n + O(1)$ CNOTs and has at least $3n$ depth.
\end{theorem}
\begin{proof}
  This proof is similar to the proof of Theorem~\ref{thm:no-go-class-i}, 
  assume that $A_j = \CNOT^{1 \to 2}$ is the only CNOT applying on qubits $1$ and $2$. We have assumed the direction without loose of generality as the other direction could be proved similarly. Now, the conjugation of $Z_1$ (acting on qubit $1$) by $A^\dagger_1 \dots A^\dagger_{j-1}$ that we call it $U_1$ must be nontrivial on qubit $1$. Using Lemma~\ref{lem:no-trivial-commutation}, the conjugation of $U_1$ by $T$ must act nontrivial on $n$. On the other hand the conjugation of $U_1$ under $A_j\dots A_1$ is equal to the conjugation of $Z_1$ under $A_j$ which is trivial on $2$ and will be trivial on any $i \ge 2$ as $A_j$ is the only CNOT between $1$ and $2$, which is not compatible with the assumption of acting nontrivially on $n$. This means that at least two CNOTs are necessary. We call the second CNOT as $A_k$ where $j < k$. Now, using Lemma~\ref{lem:no-move-class-i} we know that the conjugation of $U_1$ by $A_k\dots A_1$ is still nontrivial on $2$ that is not desired, so the third CNOT is necessary as well. Recursively we can conclude that a sequence made of three CNOTs is necessary. All of these arguments could be applied in the other direction, so we can conclude that at least $6n + O(1)$ CNOTs are necessary. 
\end{proof}
\subsection{Extension to Arbitrary Connectivity}

The extension to arbitrary connectivity is can be done by defining layers. Layers are used to linearizing the vertices of the graph with respect to their distance from a node. Then, the proof of Theorem~\ref{thm:no-go-class-i} and Theorem~\ref{thm:no-go-class-ii}
% TODO
 Given a connectivity graph $G = (V, E)$, and a target gate $T$ that is defined on $a,b \in V$, we can define a set of layers $L_1 \dots L_k$, defined as
\begin{equation}
  L_i = \{ v \in V \mid \mathrm{dist}(v, a) = i \}.
\end{equation}
Where $\mathrm{dist}(v, a)$ is the distance between $v$ and $a$ in $G$. Then, the whole proof could be re used, with the only difference that instead of acting nontrivially on $i$th qubit, now we care about acting nontrivially on one of the qubits in $L_i$. This means that the number of CNOTs and the depth will be $4\mathrm{dist}(a,b) + O(1)$ and $\mathrm{dist}(a,b) + O(1)$ respectively. 

% TODO more formal things
% TODO some shapes (for layers)